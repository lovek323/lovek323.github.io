\newcommand{\Date}{November 19, 2018}
\newcommand{\Title}{Bible Notes: 1 Peter 1}

\documentclass[a4paper,10pt]{article}

% Lwarp
%\usepackage[%
    %latexmk,
    %mathjax,
%]{lwarp}

% Bibliography
\usepackage[
    citepages=omit,
    fullbibrefs,
    ibidtracker=false,
    idemtracker=false,
    style=sbl,
]{biblatex}
\addbibresource{../../bib/books.bib}
\addbibresource{../../bib/commentaries.bib}
\addbibresource{../../bib/commentary-series.bib}
\addbibresource{../../bib/dictionaries-encyclopedias.bib}
\addbibresource{../../bib/general-series.bib}
\addbibresource{../../bib/journal-articles.bib}
\addbibresource{../../bib/websites.bib}
\setlength{\bibhang}{0.5in}

% Packages with arguments
\usepackage[bookmarks,hidelinks]{hyperref}
\usepackage[rm,tiny]{titlesec}
\usepackage[english]{babel}
\usepackage[margin=1in]{geometry}
\usepackage[marginal]{footmisc}

% Packages without arguments
\usepackage{afterpage}
\usepackage{booktabs}
\usepackage{fancyhdr}
\usepackage{graphicx}
\usepackage{longtable}
\usepackage{paralist}
\usepackage{ragged2e}
\usepackage{scrextend}
\usepackage{setspace}
\usepackage{times}
\usepackage{tocloft}
\usepackage{xcolor}

% Quotation marks -- British style
\usepackage[autostyle,english=british]{csquotes}
%\DeclareQuoteStyle[american]{english}
    %{\textquotedblleft}
    %[\textquotedblleft]
    %{\textquotedblright}
    %[0.05em]
    %{\textquoteleft}
    %{\textquoteright}
%\DeclareQuoteStyle[american-verse]{english}
    %{\textquotedblleft}
    %{\textquotedblright}
    %[0.05em]
    %{\textquoteleft}
    %{\textquoteright}
\DeclareQuoteStyle[british]{english}
    {\textquoteleft}
    [\textquoteleft]
    {\textquoteright}
    [0.05em]
    {\textquotedblleft}
    {\textquotedblright}
\DeclareQuoteStyle[british-verse]{english}
    {\textquoteleft}
    {\textquoteright}
    [0.05em]
    {\textquotedblleft}
    {\textquotedblright}
% British
\newcommand{\OpenQuote}{\textquoteleft}
% American
%\newcommand{\OpenQuote}{\textquotedblleft}

% Poetry
\newcommand{\VerseQuoteStyle}{\setquotestyle[british-verse]{english}}
\newcommand{\NormalQuoteStyle}{\setquotestyle[british]{english}}
\newcommand{\VerseIndent}{\hspace*{2em}}
\newcommand{\VerseIndentTwo}{\hspace*{4em}}
\newcommand{\VerseIndentFour}{\hspace*{8em}}

% Lengths
\setlength{\parindent}{0.5in}
\setlength{\RaggedRightParindent}{\parindent}

% Line spacing
\doublespacing{}

% Justification
\RaggedRight

% Footnotes
\setlength{\footnotesep}{19.86pt} % Determined by running \footnotesize \the\baselineskip
\renewcommand{\footnoterule}{\noindent\smash{\rule[3pt]{2in}{0.4pt}}\vspace{-0.5\footnotesep}}
\setlength{\footnotemargin}{0.5in}
\deffootnote[0.5in]{0pt}{0.5in}{}
\let\TempFootnote\footnote
\renewcommand{\footnote}[1]{\TempFootnote{\thefootnotemark.\enskip#1}}
\setlength{\skip\footins}{\baselineskip}

% Table of contents
\renewcommand\cftsecfont{\rm}
\renewcommand\cftsecleader{\cftdotfill{.}}
\renewcommand\cftsubsecdotsep{\cftnodots}

% Word count
%TC:newcounter fwords Words in footnotes
%TC:newcounter footnote Number of footnotes
%TC:macro \footnote [fwords]
%TC:macroword \footnote [footnote]
%TC:macro \autocite [fwords]
%TC:macroword \autocite [footnote]

\begin{document}

% Sections
\setcounter{secnumdepth}{0}
% - Section
\titlespacing*{\section}{0pt}{\baselineskip}{\baselineskip}
\titleformat*{\section}{\center\uppercase}
% - Subsection
\titlespacing*{\subsection}{0pt}{\baselineskip}{0pt}
\titleformat*{\subsection}{\center\bfseries}

% Headers & Footers
\renewcommand{\headrulewidth}{0pt}
\fancypagestyle{first}{%
    \fancyhf{}
    \fancyfoot[C]{\thepage}
}
\fancypagestyle{subsequent}{%
    \fancyhf{}
    \fancyhead[R]{\thepage}
}
\pagestyle{empty}
\pagenumbering{roman}

\mbox{}

\vfill

\begin{center}
    \Huge{\textbf{\Title}}
\end{center}


\vfill

\begin{center}
    \Date

    Jason O'Conal
\end{center}

\vfill
\newpage

\pagestyle{plain}

\mbox{}\vspace{1in} % Extra margin

\renewcommand{\contentsname}{}
\section*{CONTENTS}
\tableofcontents

\vfill
\newpage

\mbox{}\vspace{1in} % Extra margin

\printbiblist[title=ABBREVIATIONS]{abbreviations}

\vfill
\newpage

\pagestyle{subsequent}

\thispagestyle{first}

\pagenumbering{arabic}

\mbox{}\vspace{1in} % Extra margin


\section{Introduction}

\begin{quote}
    First Peter addresses an Anatolian church facing, at minimum, social
    pressure and ostracization, yet the letter's message is one of hope based on
    the foundation of Jesus Christ. These churches can rest assured that God has
    marked them out as a holy nation---a unique people group that ignores ethnic
    identity markers in favor of a unity forged through the blood of Jesus
    Christ. This new nation is designed to live out the love of God.
    \autocite{himes:2017}
\end{quote}

1 Peter is all about our identity, who we are in light of who God is. Once we
understand that, we can know how to live out that identity in a society that
doesn't share our beliefs.

\begin{quote}
    Christ is not an accessory to our identity, as if one were choosing an
    option for a car. He takes over identity so that everything else becomes an
    accessory, which is precisely what \enquote{Jesus is Lord} means.
    \autocite[8]{snodgrass:2011}
\end{quote}

\begin{quote}
    In summary, 1 Peter was written to churches struggling under harsh external
    pressure to remind them of who they are in Jesus Christ and how they should
    live.
    \autocite{himes:2017}
\end{quote}

\section{1 Peter 1:17}

\begin{quote}
    \textbf{Since you call on a Father who judges each person's work
    impartially, live out your time as foreigners here in reverent fear.} For
    you know that it was not with perishable things such as silver or gold that
    you were redeemed from the empty way of life handed down to you from your
    ancestors, but with the precious blood of Christ, a lamb without blemish or
    defect. He was chosen before the creation of the world, but was revealed in
    these last times for your sake. Through him you believe in God, who raised
    him from the dead and glorified him, and so your faith and hope are in God.
    (1 Peter 1:17--21)
\end{quote}

\section{Placement in 1 Peter}

Marshall's outline of 1 Peter places this verse in a section about \enquote{Fear
and Faith}:

\begin{itemize}
    \item 1:1--2. Opening Greeting
    \item 1:3--12. Thanksgiving
    \item 1:3--2:10. The Basic Characteristics of Christian Living
        \begin{itemize}
            \item 1:3--21. Hope and Obedience
                \begin{itemize}
                    \item 1:13. The Vigor of Hope
                    \item 1:14--16: Obedience and Holiness
                    \item \textbf{1:17--21. Fear and Faith}
                \end{itemize}
            \item 1:22--2:3. Love and Purity
            \item 2:4--10. The Spiritual House and the Chosen People
        \end{itemize}
    \item 2:11--3:12. Social Conduct
    \item 3:13--5:11. The Christian Attitude toward Hostility
    \item 5:12--14. Closing Greetings
        \autocite[Outline of 1 Peter]{marshall:1991}
\end{itemize}

1 Peter 1:17--21 speaks to (or at least hints at) the book's main themes:

\begin{itemize}
    \item We are foreigners in a strange land, but at the same time, we are a
        chosen people, members of God's family, redeemed with the precious blood
        of Christ
    \item We must live out our identity practically, in reverent fear
\end{itemize}

These verses jumped out to me because I've been wondering for a while whether
the Reformers' catchphrases \emph{sola gratia} and \emph{sola fide} have been
taken too far by pastors and theologians today. It seems to me that we've become
overzealous in our efforts to guard against a heresy of the past. In order to
absolutely guard against any conception that one is saved by works, we ignore
warnings given by Paul, James, Peter, the writer of the Hebrews and even Jesus
himself. And I think verse 17 is one of those warnings we've ignored. We don't
know where to put it because it seems to say that God will judge us based on our
work.

\section{It Matters How We Live}

\begin{quote}
    \emph{Christians are not in a position where it doesn't matter how they live
    because they believe in Christ and all will be forgiven at the last
    judgment.} On the contrary, they should live in this world filled with its
    temptations, with reverence for God in the face of judgment.
    \autocite[Fear and Faith, emphasis mine]{marshall:1991}\footnote{%
        Just so it doesn't seem like I am picking and choosing quotes to make a
        point, I'll add what Marshall goes on to say:

        \begin{quote}
            Peter fleshes out what this attitude involves. If he talked about
            God as the Father who should be reverenced because he judges his
            people, Peter now introduces a deeper motive for Christian conduct
            in the fact of redemption.
            \autocite[Fear and Faith]{marshall:1991}
        \end{quote}
    }
\end{quote}

It matters how we live. Maybe this is obvious to everyone else (or maybe my mind
works in a strange way and I'm getting hung up on something I shouldn't). But I
think we've lost track of the truth that we need to care about how we live not
only for the sake of the gospel (live lives of which people will ask questions,
be good ambassadors of Christ) but because we will be judged by the very person
in whom we put our hope.

Consider this example: I am terrified that people might find out the truth about
me, the truth about my sin. And sometimes, when people ask me questions about
things, I lie, I cover up my sin. This is not good behaviour. Nobody would say
it's good behaviour. But nobody's ever told me that I should tell the truth out
of reverent fear for God's judgement that will come. Why not? Perhaps most other
people have worked through what they think about this -- they might have very
good reasons for not bringing up this verse and others like it -- but it's not
something I've thought about and so, when I do come across it, I'm not sure what
to make of it.

Marshall concludes by saying:

\begin{quote}
    The more immediate question is whether this section plays its part in
    preparing Christians for life in the world: Does it furnish them with
    adequate internal resources? These verses inculcate attitudes of continuing
    faith and hope in God himself, which are based on what is known of God's
    will: that his people should be holy; that he, as the Father, is to be
    revered; that he purposes to redeem his people, lovingly and powerfully, in
    the sacrifice and resurrection of Jesus. These attitudes are still the basis
    for Christian confidence, based not on our own abilities or even on our
    faith but on the God in whom we trust.
    \autocite[Fear and Faith]{marshall:1991}
\end{quote}

I'm not convinced of his conclusion that our confidence is \enquote{based not on
our own abilities or even on our faith but on the God in whom we trust}. Of
course, I'm aware that this is exactly the sort of conclusion we're expected to
draw -- surely my confidence must be in God. It is only because of the Holy
Spirit living in me that I have the power to live a holy life. In this way, yes,
my confidence is in God and not my own abilities. And yet, I read these verses
in 1 Peter as saying we need to be holy (the preceding verses talk about this)
for two reasons: \begin{inparaenum}[(1)]
\item God will judge us impartially, and
\item it cost God a lot to redeem us.
\end{inparaenum}

Schreiner talks a little about this and gives a helpful example:

\begin{quote}
    Did Peter mean that believers should live reverently or in terror? Most
    commentators opt for the former since the confidence believers have in
    Christ seems to be at odds with the idea of living in a terrified state.
    Abject terror certainly does not fit with the joy and boldness of the
    Christian life. \emph{Reverence, however, can be watered down so that it
    becomes rather insipid.} Peter contemplated the final judgment, where
    believers will be assessed by their works and \emph{heaven and hell will be
    at stake}. There is a kind of fear that does not contradict confidence. A
    confident driver also possesses a healthy fear of an accident that prevents
    him from doing anything foolish. A genuine fear of judgment hinders
    believers from giving in to libertinism.
    \autocite[81, emphasis mine]{schreiner:2003}
\end{quote}

Schreiner refers to Wayne Grudem's assertion that \enquote{the phrase is better
understood to refer primarily or even exclusively to present judgment and
discipline in this life}\footnote{%
    These arguments are too technical for me to properly understand, but here is
    the argument from Grudem's commentary on 1 Peter:

    \begin{quote}
        Who judges each one impartially according to his deeds could be
        understood to refer to the future, final judgment in which believers
        will not be excluded from heaven but will be judged and rewarded
        according to their deeds in this life (as in Rom. 14:12; 1 Cor.
        3:10--15; 2 Cor. 5:10, etc.) However, the phrase is better understood to
        refer primarily or even exclusively to present judgment and discipline
        in this life, because:\begin{inparaenum}[(1)]
        \item this Greek construction (\textit{ton krinonta}, articular present
            participle) would naturally carry the sense \enquote{the one who is
            judging}; and
        \item the exhortation to \enquote{fear} would be inappropriate to
            address to Christians if the subject were final judgment, for
            Christians need have no fear of final condemnation.
        \end{inparaenum}

        A reference to God's present discipline in this life is appropriate for
        Peter elsewhere recognizes God's present activity of blessing and
        disciplining Christians (4:14, 17 [with the cognate word
        \textit{krima}]; cf. Heb. 12:5--11; Matt. 6:12).
        \autocite[86]{grudem:2009}
    \end{quote}
}. I'm not even sure what it would mean for God to judge us in this life --
wouldn't that judgement still be potentially bad?

Grudem says that \enquote{the exhortation to \enquote{fear} would be
inappropriate to address to Christians if the subject were final judgment, for
Christians need have no fear of final condemnation}
\autocite[86]{grudem:2009}. At the risk of being wrong (I never quite know when
this applies), I think Grudem is begging the question (by which I mean assuming
the conclusion in the premise) when answering the question of whether Peter is
referring to the final judgement, Grudem says that, no, he is referring to
discipline in this life \emph{because} \enquote{Christians need have no fear of
final condemnation} (though I think there is a jump here from fearing the final
\enquote{judgement} to fearing final \enquote{condemnation}). Grudem doesn't
seem to address the possibility that Peter is saying that just because you call
on the Father doesn't mean you don't need to fear his judgement.

\section{The Final Judgement}

Schreiner gives a number of verses that show how pervasive \enquote{judgement
according to works} is in the Bible:

\VerseQuoteStyle
\begin{quote}
    Repay them for their deeds\\
    \VerseIndent and for their evil work;\\
    repay them for what their hands have done\\
    \VerseIndent and bring back on them what they deserve. (Psalm 28:4)\\
\end{quote}

\begin{quote}
    \VerseIndent \enquote{and with you, Lord, is unfailing love};\\
    and, \enquote{You reward everyone\\
    \VerseIndent according to what they have done.} (Psalm 62:12)\\
\end{quote}

\begin{quote}
    If you say, \enquote{But we knew nothing about this,}\\
    \VerseIndent does not he who weighs the heart perceive it?\\
    Does not he who guards your life know it?\\
    \VerseIndent Will he not repay everyone according to what they have done?
    (Proverbs 24:12)\\
\end{quote}

\begin{quote}
    \enquote{I the \textsc{Lord} search the heart,\\
    \VerseIndent and examine the mind,\\
    to reward each person according to their conduct,\\
    \VerseIndent according to what their deeds deserve.}\\
\end{quote}
\NormalQuoteStyle

\begin{quote}
    They themselves will be enslaved by many nations and great kings; I will
    repay them according to their deeds and the work of their hands. (Jeremiah
    25:14)
\end{quote}

\begin{quote}
    Great are your purposes and mighty are your deeds. Your eyes are open to the
    ways of all mankind; you reward each person according to their conduct and
    as their deeds deserve. (Jeremiah 32:19)
\end{quote}

\begin{quote}
    \enquote{Before your eyes I will repay Babylon and all who live in Babylon
    for all the wrong they have done in Zion,} declares the \textsc{Lord}.
\end{quote}

\begin{quote}
    \enquote{Yet you Israelites say, \enquote{The way of the Lord is not just.}
    But I will judge each of you according to your own ways.}
\end{quote}

\begin{quote}
    For the Son of Man is going to come in his Father's glory with his angels,
    and then he will reward each person according to what they have done.
    (Matthew 16:27)
\end{quote}

\begin{quote}
    God \enquote{will repay each person according to what they have done.}
    (Romans 2:6)
\end{quote}

\begin{quote}
    For God does not show favoritism. (Romans 2:11)
\end{quote}

\begin{quote}
    A person is not a Jew who is one only outwardly, nor is circumcision merely
    outward and physical. No, a person is a Jew who is one inwardly; and
    circumcision is of the heart, by the Spirit, not by the written code. Such a
    person's praise is not from other people, but from God. (Romans 2:28--29)
\end{quote}

\begin{quote}
    So then, each of us will give an account of ourselves to God. (Romans 14:12)
\end{quote}

\begin{quote}
    Their work will be shown for what it is, because the Day will bring it to
    light. It will be revealed with fire, and the fire will test the quality of
    each person's work. (1 Corinthians 3:13)
\end{quote}

\begin{quote}
    For we must all appear before the judgment seat of Christ, so that each of
    us may receive what is due us for the things done while in the body, whether
    good or bad. (2 Corinthians 5:10)
\end{quote}

\begin{quote}
    Alexander the metalworker did me a great deal of harm. The Lord will repay
    him for what he has done. (2 Timothy 4:14)
\end{quote}

\begin{quote}
    I will strike her children dead. Then all the churches will know that I am
    he who searches hearts and minds, and I will repay each of you according to
    your deeds. (Revelation 2:23)
\end{quote}

\begin{quote}
    And I saw the dead, great and small, standing before the throne, and books
    were opened. Another book was opened, which is the book of life. The dead
    were judged according to what they had done as recorded in the books. The
    sea gave up the dead that were in it, and death and Hades gave up the dead
    that were in them, and each person was judged according to what they had
    done. (Revelation 20:12--13)
\end{quote}

\begin{quote}
    \enquote{Blessed are those who was their robes, that they may have the right
    to the tree of life and may go through the gates into the city.} (Revelation
    22:14)
\end{quote}

Schreiner says that \enquote{no dichotomy exists between judgment according to
works and God's grace} and that \enquote{the fear of judgment still plays a role
in the Christian life} and I tend to agree.

\begin{quote}
    \emph{Third, no dichotomy exists between judgment according to works and
    grace.} Good works are evidence that God has truly begotten (1 Pet 1:3) a
    person. Perhaps Peter used the singular \enquote{work} to summarize the
    lives of believers as a whole. peter reminded his readers that God is an
    \enquote{impartial} judge who does not reward people as one who plays
    favorites (cf. Acts 10:34; Rom 2:11; Eph 6:9; Col 3:25). Fourth, the fear of
    judgment still plays a role in the Christian life. Paul himself realized
    that he would be damned if he did not live the message proclaimed to others
    (1 Cor 9:24--27). Such a recognition inspires him to live faithfully; it
    does not paralyze him with fear. Paul himself taught that genuine faith
    always manifests itself in works (cf. Gal 5:21; 1 Cor 6:9--11).
    \autocite[83]{schreiner:2003}
\end{quote}

His final statement that the recognition that God will judge us doesn't paralyse
us with fear, but instead inspires us to live faithfully makes sense to me.

In the fourth book of his \textit{Christian Origins and the Question of God}
series, \textit{Paul and the Faithfulness of God}, N.T. Wright addresses this
issue in a discussion on Romans 2:1--6.

\begin{quote}
    You, therefore, have no excuse, you who pass judgment on someone else, for
    at whatever point you judge another, you are condemning yourself, because
    you who pass judgment do the same things. Now we know that God's judgment
    against those who do such things is based on truth. So when you, a mere
    human being, pass judgment on them and yet do the same things, do you think
    you will escape God's judgment? Or do you show contempt for the riches of
    his kindness, forbearance and patience, not realizing that God's kindness is
    intended to lead you to repentance? (Romans 2:1--6)
\end{quote}

\begin{quote}
    This opening paragraph (verses 1--3), springing the trap on the
    self-satisfied moralist, simply assumes a future judgment. The second
    (verses 4--6), warning against taking the divine kindness and forbearance
    for granted (these themselves are a familiar second-temple theme, an
    explanation for why judgment is delayed), simply assumes that there will be
    a coming \enquote{day of anger}, \enquote{the day when God's just judgment
    will be unveiled} (verse 5) \dots And part of this future unveiling of the
    creator's right and proper decision will be, as in verse 6, the classic
    principle: \enquote{he will repay everyone according to their works.} This
    uncontroversial maxim goes back at least to the Psalms and Proverbs, and is
    echoed in many strands of later Jewish thought. It can hardly be though un-
    or sub-Christian, since it appears in one form or another not only in Paul
    but in several other strands of the New Testament. \emph{One could only deny
    its validity if, with some late-modern trends, one were to convert the quite
    proper doctrine of \enquote{justification by faith} into its modernist
    parody, that of a God who shrugs his shoulders over human behaviour and
    \enquote{tolerates} anything and everything.}
    \autocite[1086f]{wright:2013}
\end{quote}

It seems clear to me that there is some sense in which we need to fear God's
coming judgement, that at this judgement what we have done (or works) will
matter. Wright's description of how many of us understand the doctrine of
\enquote{justification by faith} as a \enquote{modernist parody, that of a God
who shrugs his shoulders over human behaviour and \enquote{tolerates} anything
and everything} seems apt. I'm not sure how this fits with the rest of the
Bible. I am at least sure that acknowledging the reality and gravity of the
coming judgement means I don't have to explain away what seems like the clear
teaching of 1 Peter 1:17. My working hypothesis is that we must work on our
faith if we don't want to lose it, we must cooperate with the Holy Spirit to do
good and to avoid evil.

Thinking about these things has been helpful to me in my daily walk. I am
encouraged to continue to pursue holiness and obedience to the Lord Jesus.

\newpage

\thispagestyle{first}

\mbox{}\vspace{1in} % Extra margin

\printbibliography[title={BIBLIOGRAPHY}]

\end{document}

