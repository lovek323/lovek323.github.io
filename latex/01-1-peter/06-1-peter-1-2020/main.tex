\newcommand{\Date}{March 1, 2020}
\newcommand{\Title}{Teaching: 1 Peter 1}

\documentclass[a4paper,12pt]{article}

% Lwarp
\usepackage[%
    latexmk,
    mathjax,
]{lwarp}

% Bibliography
\usepackage[style=sbl,fullbibrefs]{biblatex}
\addbibresource{../../bib/books.bib}
\addbibresource{../../bib/commentaries.bib}
\addbibresource{../../bib/commentary-series.bib}
\addbibresource{../../bib/dictionaries-encyclopedias.bib}
\addbibresource{../../bib/general-series.bib}
\addbibresource{../../bib/journal-articles.bib}
\addbibresource{../../bib/websites.bib}

\usepackage[bookmarks,hidelinks]{hyperref}
\usepackage[english]{babel}
\usepackage[margin=1in]{geometry}
\usepackage[multiple]{footmisc}

\usepackage{graphicx}
\usepackage{paralist}
\usepackage{times}

% Quotes
\usepackage[autostyle,english=british]{csquotes}
\DeclareQuoteStyle[american]{english}
    {\textquotedblleft}
    [\textquotedblleft]
    {\textquotedblright}
    [0.05em]
    {\textquoteleft}
    {\textquoteright}
\DeclareQuoteStyle[american-verse]{english}
    {\textquotedblleft}
    {\textquotedblright}
    [0.05em]
    {\textquoteleft}
    {\textquoteright}
\DeclareQuoteStyle[british]{english}
    {\textquoteleft}
    [\textquoteleft]
    {\textquoteright}
    [0.05em]
    {\textquotedblleft}
    {\textquotedblright}
\DeclareQuoteStyle[british-verse]{english}
    {\textquoteleft}
    {\textquoteright}
    [0.05em]
    {\textquotedblleft}
    {\textquotedblright}
% British
\newcommand{\OpenQuote}{\textquoteleft}
% American
% \newcommand{\OpenQuote}{\textquotedblleft}

% Lengths
\setlength{\parindent}{0pt}
\setlength{\parskip}{0.5\baselineskip}

% Poetry
\newcommand{\VerseQuoteStyle}{\setquotestyle[british-verse]{english}}
\newcommand{\NormalQuoteStyle}{\setquotestyle[british]{english}}
\newcommand{\VerseIndent}{\hspace*{2em}}
\newcommand{\VerseIndentTwo}{\hspace*{4em}}
\newcommand{\VerseIndentFour}{\hspace*{8em}}

\title{\Title}
\author{Jason O'Conal}
\date{\Date}

\begin{document}
\maketitle

\tableofcontents

\printbiblist{abbreviations}


\section{What did it mean to be a follower of Jesus?}

\subsection{What it meant for the Christians to whom 1 Peter was written}

Peter wrote to the \enquote{exiles of the Dispersion} (v. 1), starting by
reminding them of the mercy of God and the living hope they had in Jesus (v. 3).

It's clear from verses 6 and 7 that these Christians had been suffering for
their faith, so it might be useful to get a brief picture of what that would've
looked like. Peter Davids gives a brief but helpful summary:

\begin{quote}
    When missionaries arrived in these provinces, people listened to the gospel
    and believed. As a result their lifestyle changed. First and foremost, they
    stopped worshiping the various gods of their empire, city, trade guild, or
    family, and instead worshiped only “the God and Father of our Lord Jesus
    Christ” (1:3). This change in behavior meant that they were now viewed as
    unpatriotic (worship of the genius of the emperor was equivalent to flag
    worship in modern America), disloyal to their city (since they would not
    take part in civic ceremonies involving worship), unprofessional in their
    trade (since guild meetings usually took place in pagan temples), and haters
    of their families (family gatherings and ceremonies also took place in
    temples, and household worship was thought to hold the family together).

    \dots\ they now followed a morality different from that of their fellow
    citizens. Previously they had enjoyed drunken parties and loose sexual
    morals, but now they demonstrated self-control in their drinking, eating,
    and sexual habits. This different behavior cut them off from their former
    friends, who thought that they had become weird (4:4).

    The result of these changes in their lives was social ostracism: insults,
    abuse, rejection, shame, and likely economic persecution with the resulting
    loss of property. There is no evidence in this letter of official
    persecution, such as imprisonment or execution, but rejection, abuse,
    punishment by family leaders (owners of slaves; husbands of women) and
    perhaps occasional mob violence had certainly taken their toll. \dots\ Their
    fellow citizens thought that these believers in Jesus no longer belonged in
    their city or family and were communicating that message loud and clear.
    \autocite[122--123]{davids:2002}
\end{quote}

On a purely practical level, being a follower of Jesus in the New Testament
times meant that every aspect of life changed -- national identity, community,
work, friends, family, morality, nothing was left untouched.

\subsection{%
    We take on a history not our own, we take on a citizenship not our own
}

I often think that I've given up a lot for Christ. And, naturally, I haven't had
to suffer like these Christians, but yeah, it's cost me, it continues to cost me
as it does any follower of Jesus. Peter has a message for his readers and for
us: we are to embrace our status as strangers. Living for Jesus is going to set
us apart from others, we'll seem weird. It might even feel -- as it did for
these Christians from so long ago -- that we don't have a physical homeland. But
what Christ offers far outweighs the best that the world can offer. We gain an
identity in Christ -- Peter talks about it in more detail in 2:9--10:
\enquote{But you are a chosen race, a royal priesthood, a holy nation, God's own
people \dots\ Once you were not a people, but now you are God's people.}

\begin{quote}
    \textcolor{gray}{%
        Second, whereas the imperial cult beckoned the inhabitants of Anatolia
        with promises of becoming part of a special group---one that revolved
        around the grandness of the \textit{Pax Romana} along with the adoration
        of the emperor and the pagan gods---\textbf{1 Peter encourages its
        audience to embrace their status as \enquote{strangers}, with full
        recognition that their lifestyle would set them apart and cause others
        to stare at them with puzzled expressions} (1 Peter 2:11; 4:4). This is
        truly countercultural. As Andreas Obermann notes, \enquote{[the concept
        of a] homeland belongs to the foundational experiences of humanity.}
        \textbf{Yet Peter essentially urges his audience to embrace the state of
        social displacement without a physical homeland}, though this is more
        than offset by the social-spiritual identity gained in 1 Peter 2:9--10.
        \autocite{himes:2017}
    }
\end{quote}

Returning to chapter 1 -- and this came up as we shared last week -- in verses
10--12, Peter talks about how faith in Jesus is part of this rich history. 1
Peter is rich with citations of and allusions to the Old Testament, giving the
reader the sense of the history. In the first two verses, there are two powerful
allusions to the Old Testament: in verse 1, the phrase \enquote{exiles of the
Dispersion} would have brought to mind the Israelites' exile under the Assyrian
and Babylonian regimes and God's promises to \enquote{regather his exiled
people} found in various prophets in the Old Testament. In verse 2, Peter refers
to an event from Exodus 24 (vv. 3--8) when he mentions \enquote{obedience} and
being \enquote{sprinkled with [Jesus's] blood.}

\begin{quote}
    \textcolor{gray}{%
        The OT is cited or alluded to in 1 Peter in rich profusion. \dots\ For a
        book of only five short chapters, there is a remarkable record of
        quotation. Yet the quotations tell only a small part of the story, for 1
        Peter is also laced with allusions to the OT\@. How many there are is
        disputed, depending on the tightness of the definition of allusion:
        Osborne sees thirty-one, while Schutter finds forty-one. If one were to
        extend beyond allusions (however defined) to echoes picking up OT
        language and themes, scarcely a verse in this epistle would be exempt.
        \autocite[1015]{carson:2007}
    }
\end{quote}

\begin{quote}
    \textcolor{gray}{%
        For example, just within chapter 1 (and even this list is partial) one
        finds the Diaspora theme in v. 1, the sanctifying work of the Holy
        Spirit and \enquote{sprinkling} language in v. 2, the inheritance theme
        in v. 4 (which becomes hugely important in the Epistle to the Hebrews),
        mention of gold refined by fire in v. 7, a collection of reflections on
        what the prophets knew and did not know in 1:10–12, a lamb without
        blemish in v. 19, mention of the creation in v. 20, and so forth.
        \autocite[1015]{carson:2007}
    }
\end{quote}

\begin{quote}
    \textcolor{gray}{%
        Literally \enquote{foreigners of the diaspora.} \dots\ The word reminds
        the reader of the impact of the exile under the Assyrian and Babylonian
        regional superpowers, with countless thousands of Jews still scattered
        all over the Mediterranean world and beyond. Through the OT prophets,
        however, God promised, both before the exile and after the initial
        return from exile, that he would regather his exiled people and
        reconstitute the twelve tribes (e.g., Isaiah 11:11--12; Jeremiah
        31:8--14; Ezekiel 37:21--22; Zechariah 10:6--12).
        \autocite[1015--1016]{carson:2007}
    }
\end{quote}

\begin{quote}
    \textcolor{gray}{%
        There is widespread acknowledgment that the words \enquote{and
        sprinkling by his blood} (NIV) allude to Exodus 24:3--8.
        \autocite[1016]{carson:2007}
    }
\end{quote}

There's something very powerful in the truth that our lives are connected to
this continuous history of God's people through the ages. The prophets wrote
about something in the distant future -- they may not have known how far into
the future it would be. Peter wrote about something in the distant future -- he
didn't know how far into the future these things would happen either. We are in
the same position -- we don't know when Jesus will return, but in Christ we are
part of the story of God's dealing with people that goes right back to Adam and
Eve, in Christ we are citizens of a new nation. This is part of what it means
when we talk about our identity in Christ.

Klyne Snodgrass put this well:

\begin{quote}
    Second, our histories and our experiences have shaped us, but Christians by
    faith have adopted a history not our own to be our own. \textbf{Our true
    history is the history of Christ into whom we are grafted. His history,
    within which and to which our personal history is subsumed, is our defining
    history.} Conversion is the acceptance of a new identity, of deriving our
    identity from someone else's story, of transferring defining power from our
    personal history and self-presence to Christ's history and His presence.
    \autocite[14]{snodgrass:2011}
\end{quote}

Not only are we part of this awesome story, we have an \enquote{imperishable,
undefiled, and unfading} \enquote{inheritance}, we have a certain hope from a
God whose \enquote{word \dots\ endures forever}.

\subsection{Not an add-on, the central defining narrative of our lives}

Becoming a Christian in the early days meant a radical shift in one's identity,
how one thought about one's place in the world, and, of course, that's true for
us today. Here's another helpful quote from Snodgrass:

\begin{quote}
    The failure to focus on identity has created enormous problems. The gospel
    in our time is an unimportant item in people's lives. It has been presented
    as a minor attachment to people, one that barely touches their identity.
    Christ, however, is not an add-on to an existing identity; He seeks to
    remove one's identity. \dots\ Christ is not an accessory to our identity, as
    if one were choosing an option for a car. He takes over our identity so that
    everything else becomes an accessory, which is precisely what \enquote{Jesus
    is Lord} means. It is the opposite of a cheap form of a gospel without
    demand and without content, as if faith were a short transaction, a prayer,
    a decision to get security taken care of in order to go to heaven.
    \autocite[8]{snodgrass:2011}
\end{quote}

\section{How should we live in light of this?}

\subsection{What does \enquote{be holy as [God is] holy} mean?}

As Snodgrass says, when we say \enquote{Jesus is Lord}, it is more than simply
\enquote{a short transaction}, it's a statement with content. Peter talks about
how to live as Christians in a hostile society. He fleshes this out more in the
remainder of the letter, but in chapter 1 verses 13--16 are particularly
relevant.

I'm going to sit in these verses for a few minutes because I think they're often
misinterpreted. I know I've misinterpreted these verses most of my life, but I
read something this week that opened my eyes to see things in a new way.

In verse 16, \enquote{for it is written, \enquote{You shall be holy, for I am
holy}}, Peter's quoting from Leviticus 19:2 which is part of what's called the
\enquote{holiness code}. In Leviticus 17--26, we read about the various ways the
Israelites could be considered \textit{unholy}, \enquote{cut off from [their]
people}. Slaughtering animals outside the camp, eating blood, inappropriate
sexual relations; it talks about honouring one's mother and father, keeping the
sabbath, not worshipping idols, and more. Then there's the day of atonement that
would restore the ritual purity and holiness of the people when they inevitably
didn't live up to God's requirements. It talks about the year of jubilee
(25:8ff) that began on the day of atonement where liberty would be proclaimed
throughout the land to all its inhabitants (25:10). Finally, in chapter 26, we
learn of the great rewards of obedience, where among other blessings, God says,
\enquote{I will look with favour upon you \dots\ I will place my dwelling in
your midst, and I shall not abhor you. I am the LORD your God who brought you
out of the land of Egypt, to be their slaves no more; I have broken the bars of
your yoke and made you walk erect.} (26:9a, 11--13) I will scatter you among the
nations.

In Hebrews 9 we read about the first covenant, about how the high priest, the
high point of human ritual purity, ritual holiness under the Levitical system,
was able to enter the Holy of Holies. Christ, we are told, \enquote{entered once
for all into the Holy Place, not with the blood of goats and calves, but with
his own blood, thus obtaining eternal redemption} (9:12). Hebrews 9:20 alludes
to the same verses in Exodus as 1 Peter 1:2 -- Exodus 24:3--8. Hebrews 9:22
tells us that \enquote{almost everything is purified with blood}, that
\enquote{without the shedding of blood there is no forgiveness of sins.}

Let's read Exodus 24:3--8 now. This is the covenant that God made with Israel
through Moses -- it was one the Israelites never lived up to, never fulfilled.
Because of their disobedience, they only brought on themselves the covenant
curses, and there was no hope that they would ever fulfil their obligations
under this covenant.

But -- and this is is the point I think Peter is making in verse 2 -- Jesus was
the perfectly obedient Israelite and so the rewards for obedience to the
covenant -- that is, God looking with favour on his people, dwelling \textit{in
their midst} -- are won for us by Jesus. The penalty for disobedience is
reversed, the exile of the Old Testament as a result of covenant justice or
judgement are brought back together as a new people of God. God described
himself as the one who frees his people from slavery, the one who breaks the
bars of oppression and makes them walk tall -- Jesus is the ultimate revelation
of who God is. Hebrews 9 reminds us of how Christ's blood has made us holy in a
way that nothing else could, in a permanent way -- there was one sacrifice for
all time. He's obtained eternal redemption for us. We've been purified by his
blood.

When Peter takes us through this grand story about the history of our salvation,
from Exodus through the prophets and back to Leviticus, when he tells us to
\textit{be holy as God is holy}, alongside telling us what we should
\enquote{do}, he's telling us what we \enquote{are}, what we are made to be
because of what Jesus has done. Peter is not first and foremost saying,
\enquote{Well, make sure you never do anything wrong; make sure you're
absolutely sinless because that's what God is} -- he's first saying, you
\textit{shall be holy} because of what Jesus has done. That's actually the
translation that the NRSV uses, not \enquote{Be holy}, but \enquote{You shall be
holy}. That's part of the reason Peter takes the path he takes in this chapter.
It's part of why he reminds us of the history of our faith, of how God's been at
work. He's saying that everything's pointing to this future where this salvation
will be revealed. All the disappointments, all the failures of the past have
been remedied in Jesus and we're now truly holy to the point where God is
willing to dwell not only \enquote{with} us, but \enquote{in} us. To the
sensibilities of anyone who'd only read the Old Testament, that would've seemed
absolutely nuts.

Yes, Peter's calling us to \enquote{be holy \dots\ in all [our] conduct}, but
it's all in the context of the unshakable, undeniable truth that we are holy
because Jesus's blood has made us holy. Nothing can take that away. Jesus has
been obedient to the covenant on our behalf and he has won all the benefits of
that covenant obedience for those who are in him. Here's something I had never
thought of before: we are actually completely ritually pure under the Mosaic
covenant, we're fit to enter the holy of holies, we're enjoying the benefits of
perfect obedience. There's nowhere in the world that's too holy for you if
you're in Jesus. In fact, you bring God's holiness into the world because
wherever you go, God is there in you.

\begin{quote}
    \textcolor{gray}{%
        Insofar as these God-provided sacrifices, which made people holy again
        under the stipulations of the Mosaic covenant, point forward to the
        ultimate sacrifice of Christ, they anticipate the way in which Christ by
        his death and resurrection makes people of the new covenant holy by
        dealing with their sin.
        \autocite[1018]{carson:2007}
    }
\end{quote}

The point here is not that we have to \textit{be} perfect, but that we
\textit{have been made holy by the sprinkling of Jesus's blood} and that makes
us fit for God's presence to dwell in us and it powerfully transforms us.

\begin{quote}
    \textcolor{gray}{%
        The expectation is that the gospel not only fits God's people for the
        presence of the holy God but also powerfully transforms them.
        \autocite[1018]{carson:2007}
    }
\end{quote}

And this all fits in with Peter's larger point in this first chapter --
\enquote{You shall be holy, for I am holy} -- these are the very words used of
Israel in Leviticus when God finalised the covenant with them and they're the
words used of Christians in 1 Peter. The difference between us and them is our
reality is perfect obedience to that covenant, Jesus has unlocked its blessings
for us. Christians inherit the promises of the covenant – we're the holy people
of God, it's who we are.

\subsection{Our whole moral framework has changed}

Because of this, our whole moral framework has changed. We don't do a lot of the
things other people do.

\subsection{Change in identity must lead to change in behaviour}

A change in identity when coupled with the transforming power of the Spirit will
inevitably lead to change in behaviour.

\begin{quote}
    \textcolor{gray}{%
        Some Christians have implied that people can act in ways totally
        separate from what they believe and know. No, we cannot. We are what we
        do. There is no antithesis of faith and works. Humans cannot live
        without acting. You will work. The question is whether you will work in
        accord with faith or unbelief, in accord with your Christ identity, or
        as if your identity is elsewhere. When faith is understood as identity
        in Christ, then works are not separable actions but the necessary living
        out of that identity.
        \autocite[16]{snodgrass:2011}
    }
\end{quote}

\subsection{Where Peter leaves it}

Peter encourages us to keep our conduct holy, he reminds us that we are holy
because of Jesus, he reminds us that our obedience to Jesus matters, that we
must constantly love one another. Once he's finished all of this, he quotes from
Isaiah 40, \enquote{All flesh is like grass,} he says. \enquote{The grass
withers, and the flower falls, but the word of the Lord endures forever.}
Certainly this is true, and when we look at the wider context, when we look at
Isaiah 40:1–11, it becomes this beautiful picture of the announcement of God's
promised redemption for his people. Let me read that now to you.

Peter is now the voice crying that \enquote{All flesh is like grass} and
\enquote{the word of the Lord endures forever.} This is not simply a commentary
on how enduring God's word is, it's a victory cry, one to be uttered at the very
moment that God is bringing about the promised rescue, redemption, and pardon of
his people. It's uttered at the moment that God's glory is revealed, when he
comes in power, when his word is shown to be absolutely unstoppable. Peter
reminds his readers that God always comes good on his promises. And like Peter's
original readers, we haven't seen Jesus, but we love him; we don't see him now,
but we believe in him and rejoice; we are receiving the outcome of our faith –
the salvation of our souls. Peter quotes Isaiah to say, don't forget for a
second who you've put your faith in. He will come through, he always does.

Peter's whole argument around holy living begins with Jesus, it ends with Jesus,
and it's filled with Jesus all the way through. He doesn't just tell us to be
holy, he tells us we're God's people, his children; he tells us we've inherited
this grand history that spans millennia, all of human history; he tells us that
we're the holy people where God's presence dwells, he tells us that we're
witnessing the victory of God, we're receiving the reward for our obedience
\textit{now}. It's an overwhelmingly beautiful picture of God's mercy and grace
to us, once lost sinners, now beneficiaries of the covenant, members of the
family, the holy people of God. This has completely revolutionised my thinking
around a verse I often just throw out there, \textit{Be holy as I am holy}. Wow.

\newpage

\thispagestyle{first}

\mbox{}\vspace{1in} % Extra margin

\printbibliography[title={BIBLIOGRAPHY}]

\end{document}

