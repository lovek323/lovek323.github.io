\newcommand{\Date}{February 25, 2019}
\newcommand{\Title}{Bible Notes: Amos 1}

\documentclass[a4paper,10pt]{article}

% Lwarp
%\usepackage[%
    %latexmk,
    %mathjax,
%]{lwarp}

% Bibliography
\usepackage[
    citepages=omit,
    fullbibrefs,
    ibidtracker=false,
    idemtracker=false,
    style=sbl,
]{biblatex}
\addbibresource{../../bib/books.bib}
\addbibresource{../../bib/commentaries.bib}
\addbibresource{../../bib/commentary-series.bib}
\addbibresource{../../bib/dictionaries-encyclopedias.bib}
\addbibresource{../../bib/general-series.bib}
\addbibresource{../../bib/journal-articles.bib}
\addbibresource{../../bib/websites.bib}
\setlength{\bibhang}{0.5in}

% Packages with arguments
\usepackage[bookmarks,hidelinks]{hyperref}
\usepackage[rm,tiny]{titlesec}
\usepackage[english]{babel}
\usepackage[margin=1in]{geometry}
\usepackage[marginal]{footmisc}

% Packages without arguments
\usepackage{afterpage}
\usepackage{booktabs}
\usepackage{fancyhdr}
\usepackage{graphicx}
\usepackage{longtable}
\usepackage{paralist}
\usepackage{ragged2e}
\usepackage{scrextend}
\usepackage{setspace}
\usepackage{times}
\usepackage{tocloft}
\usepackage{xcolor}

% Quotation marks -- British style
\usepackage[autostyle,english=british]{csquotes}
%\DeclareQuoteStyle[american]{english}
    %{\textquotedblleft}
    %[\textquotedblleft]
    %{\textquotedblright}
    %[0.05em]
    %{\textquoteleft}
    %{\textquoteright}
%\DeclareQuoteStyle[american-verse]{english}
    %{\textquotedblleft}
    %{\textquotedblright}
    %[0.05em]
    %{\textquoteleft}
    %{\textquoteright}
\DeclareQuoteStyle[british]{english}
    {\textquoteleft}
    [\textquoteleft]
    {\textquoteright}
    [0.05em]
    {\textquotedblleft}
    {\textquotedblright}
\DeclareQuoteStyle[british-verse]{english}
    {\textquoteleft}
    {\textquoteright}
    [0.05em]
    {\textquotedblleft}
    {\textquotedblright}
% British
\newcommand{\OpenQuote}{\textquoteleft}
% American
%\newcommand{\OpenQuote}{\textquotedblleft}

% Poetry
\newcommand{\VerseQuoteStyle}{\setquotestyle[british-verse]{english}}
\newcommand{\NormalQuoteStyle}{\setquotestyle[british]{english}}
\newcommand{\VerseIndent}{\hspace*{2em}}
\newcommand{\VerseIndentTwo}{\hspace*{4em}}
\newcommand{\VerseIndentFour}{\hspace*{8em}}

% Lengths
\setlength{\parindent}{0.5in}
\setlength{\RaggedRightParindent}{\parindent}

% Line spacing
\doublespacing{}

% Justification
\RaggedRight

% Footnotes
\setlength{\footnotesep}{19.86pt} % Determined by running \footnotesize \the\baselineskip
\renewcommand{\footnoterule}{\noindent\smash{\rule[3pt]{2in}{0.4pt}}\vspace{-0.5\footnotesep}}
\setlength{\footnotemargin}{0.5in}
\deffootnote[0.5in]{0pt}{0.5in}{}
\let\TempFootnote\footnote
\renewcommand{\footnote}[1]{\TempFootnote{\thefootnotemark.\enskip#1}}
\setlength{\skip\footins}{\baselineskip}

% Table of contents
\renewcommand\cftsecfont{\rm}
\renewcommand\cftsecleader{\cftdotfill{.}}
\renewcommand\cftsubsecdotsep{\cftnodots}

% Word count
%TC:newcounter fwords Words in footnotes
%TC:newcounter footnote Number of footnotes
%TC:macro \footnote [fwords]
%TC:macroword \footnote [footnote]
%TC:macro \autocite [fwords]
%TC:macroword \autocite [footnote]

\begin{document}

% Sections
\setcounter{secnumdepth}{0}
% - Section
\titlespacing*{\section}{0pt}{\baselineskip}{\baselineskip}
\titleformat*{\section}{\center\uppercase}
% - Subsection
\titlespacing*{\subsection}{0pt}{\baselineskip}{0pt}
\titleformat*{\subsection}{\center\bfseries}

% Headers & Footers
\renewcommand{\headrulewidth}{0pt}
\fancypagestyle{first}{%
    \fancyhf{}
    \fancyfoot[C]{\thepage}
}
\fancypagestyle{subsequent}{%
    \fancyhf{}
    \fancyhead[R]{\thepage}
}
\pagestyle{empty}
\pagenumbering{roman}

\mbox{}

\vfill

\begin{center}
    \Huge{\textbf{\Title}}
\end{center}


\vfill

\begin{center}
    \Date

    Jason O'Conal
\end{center}

\vfill
\newpage

\pagestyle{plain}

\mbox{}\vspace{1in} % Extra margin

\renewcommand{\contentsname}{}
\section*{CONTENTS}
\tableofcontents

\vfill
\newpage

\mbox{}\vspace{1in} % Extra margin

\printbiblist[title=ABBREVIATIONS]{abbreviations}

\vfill
\newpage

\pagestyle{subsequent}

\thispagestyle{first}

\pagenumbering{arabic}

\mbox{}\vspace{1in} % Extra margin


\section{Introduction}

As I was reading through chapter 1 of Amos this week, I had two questions:
\begin{inparaenum}[(1)]
    \item What does it mean when God says `for three sins \dots\ even for four, I
    will not relent'?
    \item How does the idea of collective responsibility that's so obvious in Amos
    fit with the idea of individual responsibility we see elsewhere in the
    Bible?
\end{inparaenum}

Amos 1:1--2 contains introductory remarks introducing the prophet and summarise
the book. After this, Amos 1:3--2:6 contains several pronouncements of judgement
against other nations (often known as `oracles against the nations') before
bringing the message closer to home with a denouncement of Israel in chapter 2.
These pronouncements share a common pattern:

\begin{enumerate}
    \item Specific sins
    \item Punishment
    \item Divine confirmation (sometimes omitted)
    \autocite[65]{smith:2017}
\end{enumerate}

There is a lot to learn from each of the oracles, but as I investigate my two
questions, I'll focus on the first one, an oracle against Damascus in Amos
1:3--5.


\section{Amos 1:3--5}

\VerseQuoteStyle
\begin{quote}
    This is what the \textsc{Lord} says:

    \VerseIndent\enquote{For three sins of Damascus,\\
    \VerseIndentTwo even for four, I will not relent.\\
    \VerseIndent Because she threshed Gilead\\
    \VerseIndentTwo with sledges having iron teeth,\\
    \VerseIndent I will send fire on the house of Hazael\\
    \VerseIndentTwo that will consume the fortresses of Ben-Hadad.\\
    \VerseIndent I will break down the gate of Damascus;\\
    \VerseIndentTwo I will destroy the king who is in the Valley of Aven\\
    \VerseIndent and the one who holds the scepter in Beth Eden.\\
    \VerseIndentTwo The people of Aram will go into exile to Kir,}\\
    \VerseIndentFour says the \textsc{Lord}.\\
\end{quote}
\NormalQuoteStyle


\section{The Fourth Sin}

The phrase \enquote{for three sins \dots\ even for four, I will not relent} is
repeated on each of the oracles in chapters 1 and 2 of Amos:

\begin{itemize}
    \item\enquote{For three sins of Damascus, even for four, I will not relent.}
    (1:3)
    \item\enquote{For three sins of Gaza, even for four, I will not relent.}
    (1:6)
    \item\enquote{For three sins of Tyre, even for four, I will not relent.}
    (1:9)
    \item\enquote{For three sins of Edom, even for four, I will not relent.}
    (1:11)
    \item\enquote{For three sins of Ammon, even for four, I will not relent.}
    (1:13)
    \item\enquote{For three sins of Moab, even for four, I will not relent.}
    (2:1)
    \item\enquote{For three sins of Judah, even for four, I will not relent.}
    (2:4)
    \item\enquote{For three sins of Israel, even for four, I will not relent.}
    (2:6)
\end{itemize}

When God showed himself to Moses on the mountain after Moses had destroyed the
first stone tablets, he described himself as \enquote{The \textsc{Lord}, the
\textsc{Lord}, the compassionate and gracious God, slow to anger, abounding in
love and faithfulness, maintaining love to thousands, and forgiving wickedness,
rebellion and sin. Yet he does not leave the guilty unpunished; he punishes the
children and their children for the sin of their parents to the third and fourth
generation.} (Exodus 34:6b--7)

The \enquote{for three \dots\ even for four} pattern, known as the $n / n + 1$
formula, is common throughout the Bible and other ancient
literature\autocite[See][70]{smith:2017}\footnote{%
\enquote{Lit., \enquote{three crimes \dots\ and four,} a typical n : n + 1
numerical parallelism.}
\autocite[306]{stuart:1988}

\enquote{This n : n + 1 formulation is a standard type of numerical
synonymous parallelism reflected frequently in the OT in various
combinations.}
\autocite[310]{stuart:1988}
}. It does not mean there were literally three sins that God overlooked but the
fourth was just too much. Instead, I think it is most likely a rhetorical device
used to show the multiplicity of the nations' sins and should not be taken
literally.\footnote{%
There are other interpretations:
\begin{inparaenum}[(1)]
    \item The wholly non-literal interpretation is does not fit all
    examples, so some say that the highest number should be taken
    literally.
    \item $3+4=7$, which represents completeness, so it would be the
    \textit{completeness} of a nation's sinfulness.
\end{inparaenum}
\autocite[See][70]{smith:2017}
The exact meaning of this phrase doesn't really affect my interpretation
here, but it is quite interesting.

Stuart translates 1:3 simply as \enquote{This is what Yahweh said: /
\textit{Because of the multiple crimes} of Damascus, / I will not restore
it, / Because they threshed the pregnant women of Gilead / With iron
threshing sledges.}
\autocite[303-304]{stuart:1988}
}

Once the pattern of sinfulness is established, Amos writes that \enquote{[God]
will not relent}. Are we seeing the limits of God's forgiveness, of his mercy?
Are we seeing sins that God won't forgive? Even if the nations repent?

This can be, I think, a rather controversial question, since it goes against a
theology cherished by many Christians, that God is always ready and eager to
forgive anyone who repents. \enquote{No one is beyond God's forgiveness,} we
say.

\subsection{Being Beyond Forgiveness in the Old Testament}

Amos' oracles are not like Jonah's oracles. When Jonah finally made it to
Ninevah, he preached a message we're much more comfortable with:
\enquote{Repent and you will be saved! Or else God's crushing judgement will
come!}

This is not the only place in the Old Testament where God says he won't forgive
certain sins:

\paragraph{Rejecting God's Salvation}\enquote{The rabble with them began to
crave other food, and again the Israelites started wailing and said, \enquote{If
only we had meat to eat! We remember the fish we ate in Egypt at no cost---also
the cucumbers, melons, leeks, onions and garlic. But now we have lost our
appetite; we never see anything but this manna} \dots\ Now a wind went out from
the \textsc{Lord} and drove quail in from the sea. It scattered them up to two
cubits deep all around the camp, as far as a day's walk in any direction. All
that day and night and all the next day the people went out and gathered quail.
No one gathered less than ten homers. Then they spread them out all around the
camp. But while the meat was still between their teeth and before it could be
consumed, the anger of the \textsc{Lord} burned against the people, and he
struck them with a severe plague.} (Numbers 11:4--6, 31--33)

\paragraph{Idolatry}\enquote{Make sure there is no man or woman, clan or tribe
among you today whose heart turns away from the \textsc{Lord} our God to go and
worship the gods of those nations \dots\ When such a person hears the words of
this oath and they invoke a blessing on themselves, thinking, \enquote{I will be
safe, even though I persist in going my own way,} they will bring disaster on
the watered land as well as the dry. \emph{The \textsc{Lord} will never be
willing to forgive them}; his wrath and zeal will burn against them. All the
curses written in this book will fall on them, and the \textsc{Lord} will blot
out their names from under heaven.} (Deuteronomy 29:18, 19--20)

\paragraph{Many Sins}\enquote{But then he [Jekoiakim] turned against
Nebuchadnezzar and rebelled. The \textsc{Lord} sent \dots\ raiders against him
to destroy Judah \dots. Surely these things happened to Judah according to the
\textsc{Lord}'s command, in order to remove them from his presence because of
the sins of Manasseh and all he had done, including the shedding of innocent
blood. For he had filled Jerusalem with innocent blood, \emph{and the
\textsc{Lord} was not willing to forgive}.} (2 Kings 24:1b, 2, 3--4)

This need not contradict God's forgiving character. Forgiveness is at the very
core of God's character, he longs to forgive. But, as it says in Exodus 34, God
\enquote{does not leave the guilty unpunished}. In each of these examples as
well as in Amos there is an established pattern of sinfulness and, I think, an
unwillingness to repent.

Are there limits to God's forgiveness? Yes, I think there are, but not for the
truly repentant.

Just so I don't leave you with a picture of an \enquote{Old Testament God} who
is unforgiving and driven by anger, I'll include a few of my favourite verses on
forgiveness from the Old Testament:

\begin{quote}
    You will again have compassion on us;\\
    \VerseIndent you will tread our sins underfoot\\
    \VerseIndent and hurl all our iniquities into the depths of the sea. (Micah
    7:19)\\
\end{quote}

\begin{quote}
    For as high as the heavens are above the earth,\\
    \VerseIndent so great is his love for those who fear him;\\
    as far as the east is from the west,\\
    \VerseIndent so far has he removed our transgressions from us. (Psalm 103:11--12)\\
\end{quote}

\VerseQuoteStyle
\begin{quote}
    Surely it was for my benefit\\
    \VerseIndent that I suffered such anguish.\\
    In your love you kept me\\
    \VerseIndent from the pit of destruction;\\
    you have put all my sins\\
    \VerseIndent behind your back. (Isaiah 38:17)\\
\end{quote}

\begin{quote}
    \enquote{No longer will they teach their neighbor,\\
    \VerseIndent or say to one another, \enquote{Know the \textsc{Lord},}\\
    because they will all know me,\\
    \VerseIndent from the least of them to the greatest,}\\
    \VerseIndent\VerseIndent declares the \textsc{Lord}.\\
    \enquote{For I will forgive their wickedness\\
    \VerseIndent and will remember their sins no more.} (Jeremiah 31:34)\\
\end{quote}

\begin{quote}
    \enquote{Come now, let us settle the matter,}\\
    \VerseIndent says the \textsc{Lord}.\\
    \enquote{Though your sins are like scarlet,\\
    \VerseIndent they shall be as white as snow;\\
    though they are red as crimson,\\
    \VerseIndent they shall be like wool.} (Isaiah 1:18)\\
\end{quote}
\NormalQuoteStyle

\subsection{Being Beyond Forgiveness in the New Testament}

We have seen that there are times when God will not forgive in the Old Testament
-- where there are repeated patterns of sin and a fixed unwillingness to repent.
What about in the New Testament? At times, it can seem as though the New
Testament presents a different God, as though God changed in some fundamental
way in the intertestamental years. But the more I learn about God, the more I
wrestle with the questions I have, the more I see that he truly is unchangeable.

There are a few instances that come to mind where forgiveness was not available
in the New Testament.

\paragraph{The Parable of the Unmerciful Servant} This is a powerful example:

\begin{quote}
    Then Peter came to Jesus and asked, \enquote{Lord, how many times shall I
    forgive my brother or sister who sins against me? Up to seven times?

    Jesus answered, \enquote{I tell you, not seven times, but seventy-seven
    times.}

    Therefore, the kingdom of heaven is like a king who wanted to settle
    accounts with his servants. As he began the settlement, a man who owed him
    ten thousand bags of gold was brought to him. Since he was not able to pay,
    the mater ordered that he and his wife and his children and all that he had
    be sold to repay the debt.

    At this the servant fell on his knees before him. \enquote{Be patient with
    me,} he begged, \enquote{and I will pay back everything.} The servant's
    master took pity on him, canceled the debt and let him go.

    But when that servant went out, he found one of his fellow servants who owed
    him a hundred silver coins. He grabbed him and began to choke him.
    \enquote{Pay back what you owe me!} he demanded.

    His fellow servant fell to his knees and begged him, \enquote{Be patient
    with me, and I will pay it back.}

    But he refused. Instead, he went off and had the man thrown into prison
    until he could pay the debt. When the other servants saw what had happened,
    they were outraged and went and told their master everything that had
    happened.

    Then the master called the servant in. \enquote{You wicked servant,} he
    said, \enquote{I canceled all that debt of yours because you begged me to.
    Shouldn't you have had mercy on your fellow servant just as I had on you?}
    In anger his master handed him over to the jailers to be tortured, until he
    should pay back all he owed.

    This is how my heavenly Father will treat each of you unless you forgive
    your brother or sister from your heart.} (Matthew 18:21--35)
\end{quote}

There is so much to learn from this parable. One of the things I think is clear
is that there are two systems we can live under:
\begin{inparaenum}[(1)]
    \item a system of forgiveness and mercy (God's preferred system) or
    \item a system of merciless justice.
\end{inparaenum}
It is our choice. If we forgive others (evidence of our own repentance), then we
can receive God's forgiveness. If we withhold forgiveness (which shows how
little we think we've been forgiven, how false our repentance really is), then
we cannot receive God's forgiveness, instead, we will receive what we originally
deserved: the debtor's prison from which there is no escape.

\paragraph{Sins Leading to Death} In 1 John, it talks about sins leading to
death:

\begin{quote}
    If you see any brother or sister commit a sin that does not lead to death,
    you should pray that God will give them life. I refer to those whose sin
    does not lead to death. There is a sin that leads to death. I am not saying
    that you should pray about that. All wrongdoing is sin, and there is sin
    that does not lead to death. (1 John 5:16--17)
\end{quote}

Writing on this passage in a post entitled, \enquote{Does God Always Give Second
Chances? John Says \enquote{No}}, Michael Heiser said:

\begin{quote}
    We can be certain that John has no specific sin in mind because he never
    names a sin in this passage. John is saying there may come a time when God
    has had enough of our sin, and then our time on earth is up. We cannot know
    when such a time might come---so we shouldn't be in the habit of sinning
    with impunity.
    \autocite{heiser:2019}
\end{quote}

\paragraph{Lying to the Holy Spirit} Heiser goes on to refer to the event in
Acts 5 where Ananias and Sapphira were God's judgement was swift and final:

\begin{quote}
    Now a man named Ananias, together with his wife Sapphira, also sold a piece
    of property. With his wife's full knowledge he kept back part of the money
    for himself, but brought the rest and put it at the apostles' feet.

    Then Peter said, \enquote{Ananias, how is it that Satan has so filled your
    heart that you have lied to the Holy Spirit and have kept for yourself some
    of the money you received for the land? Didn't it belong to you before it
    was sold? And after it was sold, wasn't the money at your disposal? What
    made you think of doing such a think? You have not lied just to human beings
    but to God.}

    When Ananias heard this, he fell down and died. And great fear seized all
    who heard what had happened. Then some young men came forward, wrapped up
    his body, and carried him out and buried him.

    About three hours later his wife came in, not knowing what had happened.
    Peter asked her, \enquote{Tell me, is this the price you and Ananias got for
    the land?}

    \enquote{Yes,} she said, \enquote{that is the price.}

    Peter said to her, \enquote{How could you conspire to test the Spirit of the
    Lord? Listen! The feet of the men who buried your husband are at the door,
    and they will carry you out also.}

    At that moment she fell down at his feet and died. Then the young men came
    in and, finding her dead, carried her out and buried her beside her husband.
    Great fear seized the whole church and all who heard about these events.
\end{quote}

\paragraph{The Unforgivable Sin} In the Gospel of Mark, Jesus describes an
unforgivable sin:

\begin{quote}
    Truly I tell you, people can be forgiven all their sins and every slander
    they utter, but whoever blasphemes against the Holy Spirit will never be
    forgiven; they are guilty of an eternal sin. (Mark 3:28--30; cf. Matthew
    12:31--32)
\end{quote}

There have been many debates over what these verses might mean and I don't
intend to claim to have solved it, but I think it's in line with what we see in
the rest of the Bible to say that the sin of persistent impenitence cannot be
forgiven. Jesus' next words in Matthew may add some context, he says that one
who \enquote{speaks against the Holy Spirit will not be forgiven, either in this
age or in the age to come} (Matthew 12:32b). People who are seeing God's own
plan for their rescue, for their salvation, and reject it (Mark 3:30 says,
\enquote{He said this because they were saying, \enquote{He has an evil
spirit.}}) have placed themselves beyond the possibility of forgiveness.

On the topic of the unforgivable sin Alec Motyer wrote this:

\begin{quote}
    What principle can be deduced for a God-pleasing life? The inadmissability
    of hatred nourished in the heart. If there is anything lying patently on the
    surface of Scripture as a candidate for being the unforgivable sin it is
    this, for nothing could be plainer than that in the absence of any outflow
    of forgiveness on the human level there can be no inflow of forgiveness from
    God.
    \autocite[42]{motyer:2011}
\end{quote}

Ezekiel chapter 18 is an extended treatment of the very questions I've been
asking. It talks of the choice we have: repent and turn away from our sin and
live or give up on righteousness and die.

\begin{quote}
    \enquote{But if a wicked person turns away from all the sins they have
    committed and keeps all my decrees and does what is just and right, that
    person will surely live; they will not die. None of the offenses they have
    committed will be remembered against them. Because of the righteous things
    they have done, they will live. Do I take any pleasure in the death of the
    wicked? declares the Sovereign \textsc{Lord}. Rather, am I not pleased when
    they turn from their ways and live?

    But if a righteous person turns from their righteousness and commits sin and
    does the same detestable things the wicked person does, will they live? None
    of the righteous things that person has done will be remembered. Because of
    the unfaithfulness they are guilty of and because of the sins they have
    committed, they will die.

    Yet you say, \enquote{The way of the Lord is not just.} Hear, you
    Israelites: Is my way unjust? Is it not your ways that are unjust? If a
    righteous person turns from their righteousness and commits sin, they will
    die for it; because of the sin they have committed they will die. But if a
    wicked person turns away from the wickedness they have committed and does
    what is just and right, they will save their life. Because they consider all
    the offenses they have committed and turn away from them, that person will
    surely live; they will not die. Yet the Israelites say, \enquote{The way of
    the Lord is not just.} Are my ways unjust, people of Israel? Is it not your
    ways that are unjust?} (Ezekiel 18:21--29)
\end{quote}

\paragraph{God's Patience Means Salvation (But It Won't Last Forever)}

\begin{quote}
    To us, of course, mercy and wrath appear as irreconcilable opposites, the
    one cancelling out the other. The perfect blending of both in one divine
    nature is something beyond our ability to fathom. As Amos perceived the
    character of his God, he saw that the lion-roar of condemnation and judgment
    came only when the patience of mercy had long, but vainly, waited for
    repentance and amendment of life. This is the significance of the repeated
    phrase \emph{for three transgressions \dots and for four}. On the part of
    man, the cup of sinfulness has been filled to the brim; on the part of God,
    there has been no hasty action: the first transgression well merited divine
    wrath, but mercy waited and patience watched. One way of expressing this
    truth about God is to say that He never punishes the sinner except after
    prolonged personal observation and ample opportunity for repentance.
    \autocite[30, emphasis original]{motyer:2011}
\end{quote}

In 2 Peter 3, it says:

\begin{quote}
    So then, dear friends, since you are looking forward to this, make every
    effort to be found spotless, blameless and at peace with him. Bear in mind
    that our Lord's patience means salvation, just as our dear brother Paul also
    wrote you with the wisdom that God gave him. (2 Peter 3:14--15)
\end{quote}

God is patient with sinners. His desire to be merciful is great, but it is not
all-encompassing. His patience is great, but it is not limitless. There will
come a day of the Lord when \enquote{The heavens will disappear with a roar; the
elements will be destroyed by fire, and the earth and everything done in it
will be laid bare.} There will come a day when God's patience runs out.

% ==========================================
% CORPORATE VERSUS INDIVIDUAL RESPONSIBILITY
% ==========================================

\section{Corporate versus Individual Responsibility}

It seems to me that in our modern western context (dominated philosophically by
the ideas of liberal democracy) we tend to focus almost exclusively on an
individual's responsibility for their own actions instead of any sort of
corporate or communal responsibility. I know it's my default way of thinking.
Anything that means I'll have to suffer because of someone else's actions
strikes me as inherently unfair and unjust. However, we find both individual and
corporate responsibility throughout the Bible. It's something I've thought about
a lot over the years and have never come up with a simple answer. I feel like
there's a beautiful truth about God waiting to be unveiled for us if we can see
the balance between corporate destiny and individual responsibility as it's
woven throughout the Bible. I feel like I'm just getting a glimpse of it --
there are probably a lot of layers of self-important individuality that I need
to peel away before I see it, let alone am able to explain it, but I'll share
some of the things I'm starting to see.

In Amos 1, we see God's message to the nations: Yahweh is God over all nations,
his standards of righteousness are written into human hearts and all will be
held to account for their sins against their fellow people.\footnote{%
\enquote{Amos 1:3--2:16 does not condemn the nations for sins against
\enquote{Israel} or \enquote{my people}, although this could be the general
intent in some cases. The focus is not on the ones sinned against; the
emphasis is on each nation's responsibility for its own inhumanity.}
\autocite[57]{smith:2017}
}

\begin{quote}
    The basis for the accusations against the nations is a more universal law of
    right and wrong which is based on conscience, national legal codes,
    international treaty rights, and a common sense view of right and wrong.
    \autocite[59]{smith:2017}
\end{quote}


% CORPORATE RESPONSIBILITY IN THE OLD TESTAMENT

\subsection{Corporate Responsibility in the Old Testament}

There are many instances of corporate destiny, judgement, responsibility in the
Old Testament. I'll list a few that came to mind.

\paragraph{Genesis: The Wickedness of the World and the Flood} In Genesis 6:5 it
says, \enquote{The \textsc{Lord} saw how great the wickedness of the human race
had become on the earth, and that every inclination of the thoughts of the human
heart was only evil all the time.} We all know the story: people were evil, God
hit the reset button but saved Noah and his family. The whole earth (or, at
least, entire people groups) were wiped out in judgement.

\paragraph{Exodus: The Plagues} In Exodus 7--12 we read of a litany of
plagues that God sent on the Egyptian people. Perhaps you could call this
\textit{persuasion} instead of \textit{judgement}, but it is still a case of God
dealing with an entire people group without the opportunity for individual
repentance. Pharaoh's heart was hardened, we're told, not the Egyptian people's.
Pharaoh was the one who made the call to keep the Israelites in slavery, not the
Egyptian people. And yet, the Egyptian people suffered all the plagues
culminating in the death of their firstborn sons.

\paragraph{Exodus: Sinai} In Exodus 20:5b--6, God commanded the Israelites not
to worship idols \enquote{for I, the \textsc{Lord} your God, am a jealous God,
punishing the children for the sin of the parents to the third and fourth
generation of those who hate me, but showing love to a thousand generations of
those who love me and keep my commandments}.

\paragraph{Joshua: The Conquest} In the first chapters of Joshua, we see
numerous nations destroyed, sometimes with all the inhabitants indiscriminately
wiped out. In Joshua 10:40, we read \enquote{So Joshua subdued the whole region,
including the hill country, the Negev, the western foothills and the mountain
slopes, together with all their kings. He left no survivors. He totally
destroyed all who breathed, just as the \textsc{Lord}, the God of Israel, had
commanded.} In Joshua 12:24, we're told that the Israelites conquered
\enquote{thirty-one kings in all}. This was a wholesale destruction.

\paragraph{Daniel: Repenting for the Community}

\begin{quote}
    Daniel is clear that the gracious response of Yahweh is dependent upon human
    response, since Daniel 9:4--5 refers to \enquote{those who love him and keep
    his commandments,} and Daniel 9:13 describes those who seek God's face by
    \enquote{turning from their iniquity and giving attention to his truth.} The
    divine response to this penitential act, however, is disappointing as
    Gabriel announces in Daniel 9:24 that the expected seventy-year limit to the
    destruction of Jerusalem would increase sevenfold in order to deal with sin.
    This suggests that although Daniel may express repentance, the community
    that he represents continues to struggle with sin.
    \autocite[668]{boda:1996}
\end{quote}

Daniel 9:17--19 records a prayer prayed by Daniel on behalf of all the people:

\begin{quote}
    \enquote{Now, our God, hear the prayers and petitions of your servant. For
    your sake, Lord, look with favor on your desolate sanctuary. Give ear, our
    God, and hear; open your eyes and see the desolation of the city that bears
    your Name. We do not make requests of you because we are righteous, but
    because of your great mercy. Lord, listen! Lord, forgive! Lord, hear and
    act! For your sake, my God, do not delay, because your city and your people
    bear your Name.}
\end{quote}

Daniel asks god to \enquote{hear}, \enquote{see}, \enquote{forgive},
and \enquote{act}. What does God's forgiveness look like? Restoring the city and
the people, restoring the community.

\paragraph{Amos: The Nations} As I highlighted above, Amos pronounced God's
judgement (through fire, which is a metaphor for war\footnote{%
\enquote{The only instrument identified in 1:4 is fire, but the
\enquote{breaking of the gate-bar of Damascus in 1:5} indicates defeat by a
military force. The use of \enquote{fire} in 1:14 or 2:2 is in a military
context and indicates that fire may be a metaphor for war rather than
something like lightning or the great fire seen in the vision in amos 7:4}
\autocite[76]{smith:2017}
}) of various nations
surrounding Israel.

\paragraph{Other Prophets: The Babylonian Exile} The Babylonian exile is
predicted and recorded the Babylonian exile. Jerusalem, the site of God's holy
temple (in which, according to Ezekiel, God no longer dwelt), was destroyed and
many of the people were dispersed into other lands. God's judgement affected all
parts of Israelite society.

Several questions arise as a result of a study like this:

\begin{itemize}
    \item How can God do this?
    \item Why weren't they given the opportunity to repent?
    \item Was there really not even a single innocent person?
    \item To quote Abraham in Genesis 18, \enquote{Far be it from you to do such
    a thing---to kill the righteous with the wicked, treating the righteous
    and the wicked alike. Far be it from you! Will not the Judge of all the
    earth do right?}
\end{itemize}

A lot of thought has been given to these questions throughout history. Various
answers have been proposed, some more convincing than others. I'm not going to
go into them here. My point is simply that God dealing with people corporately
is a thread that runs through the Old Testament and, as I aim to show, through
the New Testament as well.


% CORPORATE RESPONSIBILITY IN THE NEW TESTAMENT

\subsection{Corporate Responsibility in the New Testament}

My natural inclination, my gut instinct, is that God shifted from dealing with
people corporately in the Old Testament to individually in the New Testament,
but as I said before, I think this is mainly due to the lens through which I
view things, not necessarily what the Bible teaches.

\paragraph{The Church} I think the most powerful example of corporate destiny
and responsibility is the Church. I use a capital letter \textit{C} because I am
talking about the Church universal, not a local church.

Consider what Paul wrote in Romans 5:

\begin{quote}
    Therefore, just as sin entered the world through one man, and death through
    sin, and in this way death came to all people, because all sinned---

    To be sure, sin was in the world before the law was given, but sin is not
    charged against anyone's account where there is no law. Nevertheless, death
    reigned from the time of Adam to the time of Moses, even over those who did
    not sin by breaking a command, as did Adam, who is a pattern of the one to
    come.

    But the gift is not like the trespass. For if the many died by the trespass
    of one man, how much more did God's grace and the gift that came by the
    grace of the one man, Jesus Christ, overflow to many! Nor can the gift of
    God be compared with the result of one man's sin: The judgment followed one
    sin and brought condemnation, but the gift followed many trespasses and
    brought justification. For if, by the trespass of the one man, death reigned
    through that one man, how much more will those who receive God's abundant
    provision of grace and of the gift of righteousness reign in life through
    the one man, Jesus Christ!

    Consequently, just as one trespass resulted in condemnation for all people,
    so also one righteous act resulted in justification and life for all people.
    For just as through the disobedience of the one man the many were made
    sinners, so also through the obedience of one man the many will be made
    righteous.

    The law was brought in so that the trespass might increase. But where sin
    increased, grace increased all the more, so that, just as sin reigned in
    death, so also grace might reign through righteousness to bring eternal life
    through Jesus Christ our Lord. (Romans 5:12--20)
\end{quote}

Earlier I said that the idea that I would suffer because of someone else's
actions seemed unfair and unjust. After reading these verses from Romans 5 I
trust it's clear how incompatible that idea is with what the Bible teaches. The
whole doctrine of original sin (and that of salvation) is bound up in the idea
that a person can be born into a situation outside of their control that
determines their future. But the glorious message of Christianity is that it
goes both ways: we are born as children of Adam, but we can be reborn as
children of God; we are all condemned in Adam but we can be justified and given
life in Jesus.

When a person becomes a Christian they gain a new identity not as an individual
saint but as a member of God's family. One of the best descriptions of a
Christian in the Bible is that we are \enquote{in Christ}. In Galatians Paul
writes that \enquote{There is neither Jew nor Gentile, neither slave nor free,
nor is there male and female, for you are all one in Christ Jesus. If you belong
to Christ, then you are Abraham's seed, and heirs according to the promise.}
(Galatians 3:26--29) No Christian is saved simply as an individual Christian.
We are saved because we are in Christ, we belong to him, we are Abraham's seed,
heirs according to the promise. In Colossians 3:3 it says \enquote{For you died,
and your life is now hidden with Christ in God.}

Near the end of his letter to the Ephesians, Paul writes that Jesus
\enquote{loved the church and gave himself up for her to make her holy,
cleansing her by the washing with water through the word, and to present her to
himself as a radiant church, without stain or winkle or any other blemish, but
holy and blameless} (Ephesians 5:25--27). In 2 Corinthians 11 Paul shares his
jealousy for the church: \enquote{I am jealous for you with a godly jealousy. I
promised you to one husband, to Christ, so that I might present you as a pure
virgin to him.} (2 Corinthians 11:2)

What is salvation? It is rescue from sin and the powers of the world, adoption
into God's family, living in Christ, under his rule, in his kingdom. We aren't
saved individually, we're saved as part of the Church, the body and bride of
Christ. Jesus died for his Church. We leave our old allegiance to sin and its
powers and take up a new allegiance to Christ and his Church.

I'll give a few other examples of corporate responsibility in the New Testament.
I won't go into any of them in any great detail, but hopefully it'll be food for
thought.

\paragraph{Acts: Baptising Whole Families} \enquote{At that hour of the night
the jailer took them and washed their wounds; then immediately he and all his
household were baptized.} (Acts 16:33) This has always intrigued me and often
troubled me. I don't know how to interpret it, but it is at least a microcosm of
the nation-wide accountability we see in Amos.

\paragraph{Romans: Humanity Believed the Lie} In Romans 1:18--32, Paul writes
about how humanity rejected God:

\begin{quote}
    $^{18}$ The wrath of God is being revealed from heaven against all the
    godlessness and wickedness of people, who suppress the truth by their
    wickedness, $^{19}$ since what may be made known about God is plain to them,
    because God has made it plain to them. $^{20}$ For since the creation of the
    world God's invisible qualities---his eternal power and divine nature---have
    been clearly seen, being understood from what has been made, so that people
    are without excuse.

    $^{21}$ For although they knew God, they neither glorified him as God nor
    gave thanks to him, but their thinking became futile and their foolish
    hearts were darkened. $^{22}$ Although they claim to be wise, they became
    fools $^{23}$ and exchanged the glory of the immortal God for images made to
    look like a mortal human being and birds and animals and reptiles.

    $^{24}$ Therefore God gave them over in the sinful desires of their hearts
    to sexual impurity for the degrading of their bodies with one another.
    $^{25}$ They exchanged the truth about God for a lie, and worshiped and
    served created things rather than the Creator---who is forever praised.
    Amen.

    $^{26}$ Because of this, God gave them over to shameful lusts. Even their
    women exchanged natural sexual relations for unnatural ones. $^{27}$ In the
    same way the men also abandoned natural relations with women and were
    inflamed with lust for one another. Men committed shameful acts with other
    men, and received in themselves the due penalty for their error.

    $^{28}$ Furthermore, just as they did not think it worthwhile to retain the
    knowledge of God, so God gave them over to a depraved mind, so that they do
    what ought not to be done. $^{29}$ They have become filled with every kind
    of wickedness, evil, greed, and depravity. They are full of envy, murder,
    strife, deceit and malice. They are gossips, $^{30}$ slanderers, God-haters,
    insolent, arrogant and boastful; they invent ways of doing evil; they
    disobey their parents; $^{31}$ they have no understanding, no fidelity, no
    love, no mercy. $^{32}$ Although they know God's righteous decree that those
    who do such things deserve death, they not only continue to do these very
    things but also approve of those who practice them.
\end{quote}

Let me highlight a few points in these verses from Romans:

\begin{enumerate}
    \item God's wrath is revealed against all the evil in the world (v. 18)
    \item People's hearts had been darkened (v. 21)
    \item God gave people over to sin (vv. 24, 26, 28)
\end{enumerate}

In these verses, Paul seems to be addressing both individual and corporate
responsibility. He addresses a possible objection that a just God could not be
angry with those who didn't know better by saying that everyone knew what was
right (vv. 19--21, 25, 28, 32). But he also writes about humanity in general:
all of humanity\footnote{%
I think this even includes Israel. Paul wrote in Romans 3:9 \enquote{What
shall we conclude then? Do we have any advantage? Not at all! For we have
already made the charge that Jews and Gentiles alike are under the power of
sin.} And in 3:22--23 \enquote{There is no difference between Jew and
Gentile, for all have sinned and fall short of the glory of God.}
} had \enquote{exchanged the truth about God for a lie} and thus been given over
to sin. Was there not a single righteous person? Perhaps not. But I think this
is an example of God dealing not with individuals but with entire peoples.

In Romans 3:23, Paul wrote that \enquote{all have sinned and fall short of the
glory of God}, which is often used as a way to reconcile the primacy of the
individual in our contemporary sense of justice and God's indiscriminate wrath
against humankind. \enquote{We all deserve to go to hell because we've all
sinned.} And this is true, it makes sense of a lot, but I wonder if there's not
some corporate aspect to what Paul is saying here. In Romans 3:24, Paul goes on
to say that \enquote{all are justified freely by grace through the redemption
that came by Christ Jesus}. We've been redeemed out of the kingdom of sin into
the kingdom of God. We are justified by grace \emph{through} the redemption that
came by Christ Jesus.

\paragraph{Revelation: Letters to the Churches} Revelation 2--3 includes seven
letters to seven churches in Asia:
\begin{inparaenum}[(1)]
    \item The church in Ephesus, 2:1--7
    \item The church in Smyrna, 2:8--11
    \item The church in Pergamum, 2:12--17
    \item The church in Thyatira, 2:18--29
    \item The church in Sardis, 3:1--6
    \item The church in Philadelphia, 3:7--13
    \item The church in Laodicea, 3:14--22.
\end{inparaenum}

Jesus has a message for each of the churches. For the church in Ephesus, he says
\enquote{You have forsaken the love you had at first. Consider how far you have
fallen! Repent and do the things you did at first. If you do not repent, I will
come to you and remove your lampstand from its place.} (Revelation 2:4b--5)
Earlier, in Revelation 1:11, we're told \enquote{the seven lampstands are the
seven churches}. In a message to the whole local body of believers, Jesus says
that if they don't repent they won't be a church anymore! Can you imagine the
heckles that would be raised were anyone to suggest something like this to a
whole church these days? Certainly, there are many points on this related to
individual believers that could be discussed, but there is something worth
considering about the collective nature of Jesus' messages here.


% INDIVIDUAL RESPONSIBILITY IN THE OLD TESTAMENT

\subsection{Individual Responsibility in the Old Testament}

Having looked at corporate responsibility in the Old and New Testaments and (I
hope) having shown that there is no fundamental shift in the way God deals with
humanity (God doesn't stop dealing with us corporately, we are just given the
option to join a new corporation!), I'll move on to addressing the other side of
the continuity of the Bible: individual responsibility in the Old and New
Testaments. I'll start as I did earlier, with the Old Testament.

You might think, as I did, that individual responsibility isn't \emph{really}
present in the Old Testament, that God was primarily (almost exclusively)
concerned with Israel as a whole and not people individually. Or, you may have
always been able to see what I'm now only starting to come to grips with: though
God deals with people corporately in both the Old and New Testaments, he is (and
always has been) also supremely concerned about the inner quality of repentance
and personal devotion of the individual.

\paragraph{Genesis: Cain} In Genesis 4:10--12, we read, \enquote{Then the
\textsc{Lord} said \enquote{What have you done? Listen! Your brother's blood
cries out to me from the ground. Now you are under a curse and driven from the
ground, which opened its mouth to receive your brother's blood from your hand.
When you work the ground, it will no longer yield its crops for you. You will be
a restless wanderer on the earth.}} Cain sinned by killing his brother Abel.
Cain himself was punished by God for this sin. Taken at face value, this seems
to be an instance of individual sin and individual responsibility.

\paragraph{Exodus: Moses' Sin} In Numbers 20, we are told the Israelites had no
water and they argued with Moses, complaining about their lives in the
wilderness. God told Moses to take his staff and speak to the rock and it would
pour out water. Instead of simply speaking to the rock, Moses struck it twice.
God said \enquote{Because you did not trust in me enough to honor me as holy in
the sight of the Israelites, you will not bring this community into the land I
give them.} Moses was judged for his sin. Many others got to go to the promised
land, but he didn't.

\paragraph{Joshua: Rahab} In Joshua 2, we read of a prostitute called Rahab who
protected two Israelite spies. Rahab made the spies swear that she would be
saved when they took the land: \enquote{\enquote{Our lives for your lives!} the
men assured her. \enquote{If you don't tell what we are doing, we will treat you
kindly and faithfully when the \textsc{Lord} gives us the land.}} (Joshua 2:14)
Rahab and her family were saved out of all the people of Jericho. \enquote{But
Joshua spared Rahab the prostitute, with her family and all who belonged to her,
because she hid the men Joshua had sent as spies to Jericho -- and she lives
among the Israelites to this day.} (Joshua 6:25) Perhaps the most striking
example of corporate responsibility in the Old Testament, the conquest of
Canaan, contains within it an example of individual responsibility -- Rahab was
saved because she saw the power of God and the coming judgement and responded
appropriately.\footnote{%
    Rahab was included in the list of the great faithful in Hebrews 11:31. Rahab
    is called \enquote{righteous} in James 2:25.
}

\paragraph{Ruth} Ruth chose to follow Naomi and serve Yahweh:
\enquote{\enquote{Where you go I will go, and where you stay I will stay. Your
people will be my people and your God my God. Where you die I will die, and
there I will be buried. May the \textsc{Lord} deal with me ever so severely, if
even death separates you and me}} (Ruth 1:16b--17). Ruth was a Moabitess. Moab
was no friend to Israel. In fact, Amos announced God's judgement against Moab at
the beginning of chapter 2. Ruth went against her own community and allied
herself to the community of God.

\paragraph{Deuteronomy: Each Will Be Judged} Cain, Moses, Rahab and Ruth are all
examples of individuals being held personally responsible for their own actions.
There are, of course, many more examples that could be included. They all
support the principle we see in Deuteronomy 24:16, that \enquote{Parents are not
to be put to death for their children, nor children put to death for their
parents; each will die for their own sin.} (Deuteronomy 24:16)

\paragraph{Ezekiel: A Possible Example}

In Ezekiel 18, Ezekiel records God asking why people are quoting a particular
proverb: \enquote{The parents eat sour grapes, / and the children's teeth are
set on edge} (Ezekiel 18:2).

Ezekiel 18:3--18 describes three men: a man, his child and his grandchild.

\begin{enumerate}
    \item A righteous man who will surely live (18:5--9)
    \item A violent son who will be put to death, whose blood will be on his own
    head (18:10--13)
    \item Another righteous man who will not die for his father's sin.
\end{enumerate}

\begin{quote}
    \OpenQuote Yet you ask, \enquote*{Why does the son not share the guilt of his
    father?} Since the son has done what is just and right and has been careful
    to keep all my decrees, he will surely live. The one who sins is the one who
    will die. The child will not share the guilt of the parent, nor will the
    parent share the guilt of the child. The righteousness of the righteous will
    be credited to them, and the wickedness of the wicked will be charged
    against them.

    \dots

    \enquote{Therefore, you Israelites, I will judge each of you according to
    your own ways, declares the Sovereign \textsc{Lord}. Repent! Turn away from
    all your offenses; then sin will not be your downfall. Rid yourselves of all
    the offenses you have committed, and get a new heart and a new spirit. Why
    will you die, people of Israel? For I take no pleasure in the death of
    anyone, declares the Sovereign \textsc{Lord}. Repent and live!}
\end{quote}

I should note that one of the articles I read while researching this topic
suggested that Ezekiel 18 is in \enquote{a corporate context, where the sin and
punishment of the nation in view}\footnote{%
\enquote{This section often is seen as a statement of individual
responsibility regarding sin and judgment, but P. Joyce rightly points out
that the arguments appear in a corporate context, where the sin and
punishment of the nation are in view.}
\autocite[470]{hayes:2012}
}. I, personally, cannot see this. Yes, the first 24 chapters of Ezekiel deal
with God's judgement of the nation, the Glory of Yahweh leaving the temple, the
capture of Jerusalem, but I see chapter 18 as both a comfort and a warning to
those who feel like they're suffering for someone else's sins. I think God might
be saying that, yes, they may \emph{suffer} for others' sins but they will be
\emph{judged} for their own sins. I don't know enough, I haven't read enough, to
draw a conclusion on whether this passage is referring to the nation not
suffering for the sins of its collective fathers or whether it really is talking
about individual responsibility for God. Perhaps I'm clouded by my own biases
toward individualism, but I lean toward the latter. It won't surprise me too
much to find that I'm wildly wrong. One possible solution I can see is that
Ezekiel is not about individual versus corporate responsibility but about
historical versus present responsibility.

This seems to be the view taken by Schaefer:

\begin{quote}
    Ezekiel amplified the ramifications of this latter verse, arguing that it
    was not the sins of the fathers but the sin of his generation that was being
    judged. \dots Ezekiel commanded them to quit hiding behind the proverb; they
    were also accountable. Instead, he had them focus on the truth that
    \enquote{The soul who sins is the one who will die} (v. 4) and \enquote{The
    righteousness of the righteous man will be credited to him, and wickedness
    of the wicked will be charged against him} (v. 20).
    \autocite[674]{schaefer:1996}
\end{quote}


% INDIVIDUAL RESPONSIBILITY IN THE NEW TESTAMENT

\subsection{Individual Responsibility in the New Testament}

We have seen corporate responsibility throughout the Bible. We have seen
individual responsibility in the Old Testament. Now it's time to look at
individual responsibility in the New Testament. I'll spend less time on
this than on the other sections because it seems to me to be the most obvious.

\paragraph{The Gospels: Jesus's Ministry} Jesus's ministry was full of personal
calls to people: \enquote{Then Jesus said to his disciples, \enquote{Whoever
wants to be my disciple must deny themselves and take up their cross and follow
me. For whoever wants to save their life will lose it, but whoever loses their
life for me will find it. What good will it be for someone to gain the whole
world, yet forfeit their soul? Or what can anyone give in exchange for their
soul? For the Son of Man is going to come in his Father's glory with his angels,
and then he will reward each person according to what they have done.}} (Matthew
16:24--27) In Matthew 7:21--23, Jesus said, \enquote{Not everyone who says to
me, \enquote{Lord, Lord} will enter the kingdom of heaven. Many will say to me
on that day, \enquote{Lord, Lord, did we not prophesy in your name and in your
name drive out demons and in your name perform many miracles?} Then I will tell
them plainly, \enquote{I never knew you. Away from me, you evildoers.}}

\paragraph{The Gospels: The Thieves on the Cross} One of the most striking
examples of a person making a personal choice to trust in Jesus comes near the
end of Luke's Gospel: \enquote{One of the criminals who hung there hurled
insults at him: \enquote{Aren't you the Messiah? Save yourself and us!} But the
other criminal rebuked him. \enquote{Don't you fear God,} he said,
\enquote{since you are under the same sentence? We are punished justly, for we
are getting what our deeds deserve. But this man has done nothing wrong.} Then
he said, \enquote{Jesus, remember me when you come into your kingdom.} Jesus
answered him, \enquote{Truly I tell you, today you will be with me in
paradise.}} (Luke 23:39--43) These two men responded to Jesus very differently.
One ended up with Jesus in paradise that very day. They (and nobody else) reaped
the rewards of their actions on earth.

\paragraph{Acts: Ananias and Sapphira} We have already looked at this story in
Acts 5. Ananias was struck down because he lied to the Holy Spirit. Shortly
afterwards, Sapphira was struck down for the same reason. They weren't struck
down for anyone else's sin, they were struck down for their own sin.

\paragraph{Romans: Each Will Be Judged} In Romans 14:12, Paul wrote, \enquote{So
then, each of us will give an account of ourselves to God.} Just before this,
Paul wrote that \enquote{For we will all stand before God's judgment seat. It is
written: \enquote{As surely as I live,} says the Lord, \enquote{every knee will
bow before me; every tongue will acknowledge God.}} Every knee will bow. Every
tongue will acknowledge God. Every single one. We'll all be held individually
responsible for our response to God.

\subsection{Conclusions on Responsibility}

I hope I have been able to show that both corporate and individual
responsibility run through the entire Bible. There are many other threads on
these topics that could be pulled. For example, how can we reconcile God's
promise in Exodus 20:5--6 that he would \enquote{[punish] the children for the
sin of the parents to the third and fourth generation} with Ezekiel 18:2 where
God chides the people for saying \enquote{The parents eat sour grapes, / and the
children's teeth are set on edge} or in Deuteronomy 24:16 where the people are
told that \enquote{parents are not to be put to death for their children}? My
intention is not to answer every possible question, but simply to pull whatever
thread seems to be before me on a given day. Today's thread has been corporate
versus individual responsibility through the two testaments.


% ==========
% CONCLUSION
% ==========

\section{Conclusion}

Earlier, I mentioned that it seems at times that God has changed, that there are
fundamental differences between God in the Old Testament and the New Testament.
Naturally, we learn more about God and we understand him in an entirely new and
deeper way through Christ, the supreme revelation of God to man. But one of the
things I'm really enjoying about reading Amos is how I am seeing more and more
of God's constancy through the ages.

He didn't go from an unforgiving God in the Old Testament to a forgiving God in
the New Testament; grace, forgiveness and mercy have always been driving forces
at the core of his being in a way that wrath never has.

He didn't go from treating humanity corporately to treating humanity
individually; individual responsibility and corporate destiny have always been
closely intertwined. The normal method of salvation that God has made available
has always been through membership of his people -- Israel in the Old Testament
and the Church (the New Israel) in the New Testament.

Sometimes as Christians we pit law and grace against each other. Atonement
through sacrifices in the Old Testament didn't put God in a person's debt as
though he had to forgive them. No, God's forgiveness has always been a work of
grace. God chooses to forgive us and call us blameless in the Old Testament and
the New. God is the same yesterday, today and for ever!

I started this section by talking about how foreign I think the concept of
corporate responsibility is to our culture, but as I've been thinking about it
I've started to see how God absolutely must deal with people corporately. I gave
the example of the Israelite conquest of Canaan; a society can become so
poisoned that it simply cannot be allowed to continue in its present form. But
on a smaller scale it is also plainly true that, if a society has bad leaders or
a large number of very evil people, it will suffer. If God's judgement against
the leaders is to overthrow their rule, it will result in significant changes to
the society. Additionally, generally speaking, bad leaders flourish because of
the attitudes of their people. No great evil ever seems to come up out of a
vacuum. On the other hand, if a society has good leaders or is filled with good
people, then everybody benefits, even the evil people! I think we see two types
of God's judgement: \begin{inparaenum}[(1)]
                        \item judgement in time, where the good prevail and the evil are punished in a
                        way that we can see this side of eternity
                        \item the final judgement, where everyone is judged according to what they have
                        done.
\end{inparaenum}

Hayes wrote that \enquote{For the Israelites, proper worship and effective
community leadership were the expected manifestations of a flourishing
relationship between God and his people.}
\autocite[471]{hayes:2012} This makes sense to me and it applies to the local
church just as much as to Israel: if we are in a community where the leaders and
most people have flourishing relationships with God and we express that
individually as well as communally (i.e., if you could describe our
\emph{community} as having a flourishing relationship with God), then I would
expect good leadership and a much happier life.

Corporate and individual responsibility don't necessarily fit neatly into these
categories. However, I think the Bible teaches the perspective that because of
his love for his Creation there is evil on this world that God does not allow to
continue\footnote{%
The way he makes this decision is not really known to us, but we're told in
Psalm 34:17 \enquote{The righteous cry out, and the \textsc{Lord} hears
them} or in Exodus 3:7, \enquote{I have indeed seen the misery of my people
in Egypt. I have heard them crying out because of their slave drivers, and I
am concerned about their suffering}
} but that the real judgement, the eternal judgement comes not in this life, not
in a way we can see, but after death. If we take this perspective, then, I think
on one level we still have a God whose temporal judgement can be unfair at the
individual level but God ultimately fulfils his promise that he won't leave the
guilty unpunished. I don't think anyone can argue that God's treatment of
humanity in this life is completely fair. I didn't get to be Adam (I'm fairly
convinced I would've fallen as well), but I was, quite literally, born this way,
yet I am still responsible and will be judged according to my own actions. I get
to live in Australia, at a time where the levels of comfort and happiness are
unparalleled through human history. Very few people alive today get that
opportunity not to mention the people throughout history. For some people, each
of the few days they live is filled with a type of sadness, pain, deep despair
that I cannot even imagine. I don't think it's disrespectful to say that God's
treatment of us is absolutely unfair, but only if we consider what happens this
side of eternity to be the final word.

I feel so ill-equipped to be writing about these things because I know that
whatever I write will be hopelessly inadequate. I hope to share some of the
things I'm noticing that feel really eye-opening. As always, it might be totally
obvious to everyone else, but I think it's something a lot of people are
missing.

\newpage

\thispagestyle{first}

\mbox{}\vspace{1in} % Extra margin

\printbibliography[title={BIBLIOGRAPHY}]

\end{document}

