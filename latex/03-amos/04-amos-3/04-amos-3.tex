\newcommand{\Date}{April 25, 2019}
\newcommand{\Title}{Bible Notes: Amos 3}

\documentclass[a4paper,10pt]{article}

% Lwarp
%\usepackage[%
    %latexmk,
    %mathjax,
%]{lwarp}

% Bibliography
\usepackage[
    citepages=omit,
    fullbibrefs,
    ibidtracker=false,
    idemtracker=false,
    style=sbl,
]{biblatex}
\addbibresource{../../bib/books.bib}
\addbibresource{../../bib/commentaries.bib}
\addbibresource{../../bib/commentary-series.bib}
\addbibresource{../../bib/dictionaries-encyclopedias.bib}
\addbibresource{../../bib/general-series.bib}
\addbibresource{../../bib/journal-articles.bib}
\addbibresource{../../bib/websites.bib}
\setlength{\bibhang}{0.5in}

% Packages with arguments
\usepackage[bookmarks,hidelinks]{hyperref}
\usepackage[rm,tiny]{titlesec}
\usepackage[english]{babel}
\usepackage[margin=1in]{geometry}
\usepackage[marginal]{footmisc}

% Packages without arguments
\usepackage{afterpage}
\usepackage{booktabs}
\usepackage{fancyhdr}
\usepackage{graphicx}
\usepackage{longtable}
\usepackage{paralist}
\usepackage{ragged2e}
\usepackage{scrextend}
\usepackage{setspace}
\usepackage{times}
\usepackage{tocloft}
\usepackage{xcolor}

% Quotation marks -- British style
\usepackage[autostyle,english=british]{csquotes}
%\DeclareQuoteStyle[american]{english}
    %{\textquotedblleft}
    %[\textquotedblleft]
    %{\textquotedblright}
    %[0.05em]
    %{\textquoteleft}
    %{\textquoteright}
%\DeclareQuoteStyle[american-verse]{english}
    %{\textquotedblleft}
    %{\textquotedblright}
    %[0.05em]
    %{\textquoteleft}
    %{\textquoteright}
\DeclareQuoteStyle[british]{english}
    {\textquoteleft}
    [\textquoteleft]
    {\textquoteright}
    [0.05em]
    {\textquotedblleft}
    {\textquotedblright}
\DeclareQuoteStyle[british-verse]{english}
    {\textquoteleft}
    {\textquoteright}
    [0.05em]
    {\textquotedblleft}
    {\textquotedblright}
% British
\newcommand{\OpenQuote}{\textquoteleft}
% American
%\newcommand{\OpenQuote}{\textquotedblleft}

% Poetry
\newcommand{\VerseQuoteStyle}{\setquotestyle[british-verse]{english}}
\newcommand{\NormalQuoteStyle}{\setquotestyle[british]{english}}
\newcommand{\VerseIndent}{\hspace*{2em}}
\newcommand{\VerseIndentTwo}{\hspace*{4em}}
\newcommand{\VerseIndentFour}{\hspace*{8em}}

% Lengths
\setlength{\parindent}{0.5in}
\setlength{\RaggedRightParindent}{\parindent}

% Line spacing
\doublespacing{}

% Justification
\RaggedRight

% Footnotes
\setlength{\footnotesep}{19.86pt} % Determined by running \footnotesize \the\baselineskip
\renewcommand{\footnoterule}{\noindent\smash{\rule[3pt]{2in}{0.4pt}}\vspace{-0.5\footnotesep}}
\setlength{\footnotemargin}{0.5in}
\deffootnote[0.5in]{0pt}{0.5in}{}
\let\TempFootnote\footnote
\renewcommand{\footnote}[1]{\TempFootnote{\thefootnotemark.\enskip#1}}
\setlength{\skip\footins}{\baselineskip}

% Table of contents
\renewcommand\cftsecfont{\rm}
\renewcommand\cftsecleader{\cftdotfill{.}}
\renewcommand\cftsubsecdotsep{\cftnodots}

% Word count
%TC:newcounter fwords Words in footnotes
%TC:newcounter footnote Number of footnotes
%TC:macro \footnote [fwords]
%TC:macroword \footnote [footnote]
%TC:macro \autocite [fwords]
%TC:macroword \autocite [footnote]

\begin{document}

% Sections
\setcounter{secnumdepth}{0}
% - Section
\titlespacing*{\section}{0pt}{\baselineskip}{\baselineskip}
\titleformat*{\section}{\center\uppercase}
% - Subsection
\titlespacing*{\subsection}{0pt}{\baselineskip}{0pt}
\titleformat*{\subsection}{\center\bfseries}

% Headers & Footers
\renewcommand{\headrulewidth}{0pt}
\fancypagestyle{first}{%
    \fancyhf{}
    \fancyfoot[C]{\thepage}
}
\fancypagestyle{subsequent}{%
    \fancyhf{}
    \fancyhead[R]{\thepage}
}
\pagestyle{empty}
\pagenumbering{roman}

\mbox{}

\vfill

\begin{center}
    \Huge{\textbf{\Title}}
\end{center}


\vfill

\begin{center}
    \Date

    Jason O'Conal
\end{center}

\vfill
\newpage

\pagestyle{plain}

\mbox{}\vspace{1in} % Extra margin

\renewcommand{\contentsname}{}
\section*{CONTENTS}
\tableofcontents

\vfill
\newpage

\mbox{}\vspace{1in} % Extra margin

\printbiblist[title=ABBREVIATIONS]{abbreviations}

\vfill
\newpage

\pagestyle{subsequent}

\thispagestyle{first}

\pagenumbering{arabic}

\mbox{}\vspace{1in} % Extra margin


\section{Introduction}

The third chapter of Amos seems to focus on dispelling the Israelites' false
sense of security. I'll touch on most parts of Amos 3, but I'll start with
verses 3--8 which I found quite confusing on my first reading.


\section{Verses 3--8}

\begin{quote}
    \textsuperscript{3} Do two walk together\\
    \VerseIndent unless they have agreed to do so?\\
    \textsuperscript{4} Does a lion roar in the thicket\\
    \VerseIndent when it has no prey?\\
    Does it growl in its den\\
    \VerseIndent when it has caught nothing?\\
    \textsuperscript{5} Does a bird swoop down to a trap on the ground\\
    \VerseIndent when no bait is there?\\
    Does a trap spring up from the ground\\
    \VerseIndent if it has not caught anything?\\
    \textsuperscript{6} When a trumpet sounds in a city,\\
    \VerseIndent do not the people tremble?\\
    When disaster comes to a city,\\
    \VerseIndent has not the \textsc{Lord} caused it?

    \textsuperscript{7} Surely the Sovereign \textsc{Lord} does nothing\\
    \VerseIndent without revealing his plan\\
    \VerseIndent to his servants the prophets.

    \textsuperscript{8} The lion has roared---\\
    \VerseIndent who will not fear?\\
    The Sovereign \textsc{Lord} has spoken---\\
    \VerseIndent who can but prophesy?
\end{quote}

What on earth do these verses mean? I couldn't work it out by just looking at
it. In certain circles, there seems to be the idea that one ought to be able to
understand the Bible on its own without any external inputs. I wasn't able to
understand these verses -- I couldn't even really come up with \textit{ideas}
about what they might mean -- without consulting what others have written
concerning them.

There were two main interpretations of these verses in the texts I read:

\begin{enumerate}
    \item Amos is driving home the idea of cause and effect: actions have
        consequences; Amos is defending his message
    \item Amos is defending his prophetic office, defending his right to speak,
        defending himself not his message
\end{enumerate}

I'll split this passage up into four sections and deal with each in turn:

\begin{enumerate}
    \item v. 3
    \item vv. 4--6
    \item v. 7
    \item vv. 6, 8
\end{enumerate}

\subsection{Verse 3: Walking Together}

This is a strange verse, isn't it? The NRSV translates it `Do two walk together
unless they have made an appointment?' which sounds quite different to my ear.
In my research, this verse was interpreted either as
\begin{inparaenum}[(1)]
\item about God and Israel, or
\item about God and Amos.
\end{inparaenum}

Motyer interprets verse 3 as being about God and Israel:

\begin{quote}
    Two people do not \enquote{keep company} except by making an arrangement.
    Just such an arrangement was made between the Lord and His people at the
    Exodus and they began to \enquote{keep company}.
    \autocite[70]{motyer:1974}
\end{quote}

Calvin, on the other hand, argues that verse 3 refers to God and his prophet,
Amos:

\begin{quote}
    Some forcibly misapply the Prophet's words, as though the meaning was, that
    God was constrained to depart from that people, because he saw that they
    were going astray so perversely after their lusts.

    \dots

    The Prophet here affirms that he speaks by God's command, as when two agree
    together, when they follow the same road; as when one meets with a chance
    companion, he asks him where he goes, and when he answers that he is going
    to a certain place, he says, I am going on the same road with you. Then Amos
    by this similitude very fitly sets forth the accordance between God and his
    Prophets.
    \autocite[204-205]{calvin+owen:1986}
\end{quote}

I will discuss this more below, but I agree with Motyer et al.\ that the idea
here is that God and Israel have agreed to \enquote{keep company}. What is the
content of this agreement? The covenant entered into by the Israelites at Mount
Sinai.

\subsection{Verses 4--6: Cause and Effect}

Verses 4--6 are central to our passage and they have been understood as fitting
into one or the other of the overarching interpretations I mentioned above:
either Amos is driving home the truthfulness of his message or Amos is defending
his own authority as God's prophet. 

\begin{quote}
    Verse 8 comes as a climax to a series of verses which proclaim that
    there is no effect without a cause.
    \autocite[75]{motyer:1974}
\end{quote}

As we saw above, Calvin reads verse 3 as about the relationship between God and
his prophet and not between God and Israel. Calvin sees a slightly different
(perhaps deeper) meaning in these \enquote{similitudes}:

\begin{itemize}
    \item v. 4: \enquote{God does not cry out for nothing by his Prophets}
    \item v. 5: \enquote{nothing happens without being foreseen by God}
    \item v. 6: \enquote{the threatenings of God would not be without effect}
        \autocite[206-207]{calvin+owen:1986}
\end{itemize}

The cause-and-effect pattern that other commentators see\footnote{%
    Niehaus sees vv. 3--6 as having a \enquote{cause-and-effect structure (vv.
    3--6)}, but he admits the that \enquote{it is also the relationship between
    the Lord and his prophet is indicated here [v. 3]}.
    \autocite[378]{mccomiskey:2009}

    Smith sees vv. 1--8 as addressing the shocking nature of his message:
    \enquote{For the nation to respond positively to the word of God concerning
    the destruction of Israel, the prophet must overcome the illogical nature of
    supposing that the God who chose Israel would now destroy her.}
    \autocite[135]{smith:2017} Smith objects to Calvin's viewpoint: \enquote{The
    significance of 3:3--8 is all too frequently centered around the prophet's
    justification of himself.} \autocite[146]{smith:2017}
} makes a lot of sense to me in a context where the prophet is trying to
convince a rebellious people that their actions do in fact have consequences not
necessarily because of who God is but more because of the very nature of things.
Calvin's conclusions are more about God (purpose, foresight, power) than about
the nature of things. I don't see the depth of meaning that Calvin does in these
passages, but they are nevertheless true and at least the first point (that
\enquote{God does not cry out for nothing by his Prophets}) is, I think, backed
up by other parts of the passage.

\subsection{Verses 6 and 8: The Appropriate Response}

In verses 6 and 8, Amos is talking about two things:
\begin{inparaenum}[(1)]
\item the appropriate response to the current message of disaster
\item the source of the coming disaster.
\end{inparaenum}

In verse 6, Amos says to the people, \enquote{You know what to do when the
trumpet sounds, right? Yes, you tremble in fear.} In verse 8, Amos says to the
people, \enquote{The lion \textit{has} roared, you should be afraid.} In verse
6, we're told that this sort of disaster is caused by God. In verse 8, we're
told that Amos' message (about the coming disaster) comes from God and that he
is just the messenger.

\subsection{Verse 7: The Privilege of Knowing God}

Verse 7 is a difficult one. We're told that God \enquote{does nothing without
revealing his plan to \dots the prophets}. When I read this, I apply my modern,
western analytical mindset and so my first thought is that Amos is saying God
tells every little detail of all of his plans to the prophets. \enquote{But what
about Jesus?} I wonder. I don't think the prophets knew the details of God's
future plan of redemption. Otherwise, as Amos himself says, he would be unable
to keep it to himself (\enquote{who can but prophesy?}, v. 8).

Perhaps, as some say, it is simply God's normal way of acting\footnote{%
    \enquote{It is God's general plan to anticipate coming events by informing
    His servants the prophets of what He intends to do.}
    \autocite[73]{motyer:1974}
} and is not intended as a statement about every single instance in the way I
naturally read it. I'm not completely satisfied by this explanation, it seems to
me to be watering down Amos' words.

Some see verses 7--8 as Amos defending his right to speak, defending his
message\footnote{%
    Commenting on verse 7: \enquote{Thus Amos' lesson on cause and effect is a
    substantiation of his role as a prophet.} \autocite[381]{mccomiskey:2009}
}. They see it as Amos reminding his audience of how God does in fact let his
prophets in on his secret plans and that Amos is God's messenger, that Amos is
speaking the truth. And this does come through in these verses, but if we read
verses 3--6 and 8a as Amos driving home the sombre reality that faces the
Israelites, then it makes sense to me for vv. 7--8a to be serving the same
purpose. In verse 7, Amos is perhaps saying, this is truly a word from the
\textsc{Lord}, this is truly a possibility and it fits with the way God usually
delivers this sort of message: he tells his prophets\footnote{%
    \enquote{It is not that Yahweh does nothing at all without telling the
    prophets; rather, he does nothing by way of covenant-lawsuit judgment
    without telling them.} \autocite[380]{mccomiskey:2009}
}. A paraphrase could be, \enquote{Surely the Sovereign \textsc{Lord} brings no
judgement against his people without revealing it through his servants the
prophets}, the fact that a prophet is delivering this message speaks to the
gravity of the situation, it should not be dismissed.

\subsection{Falling Out of God's Favour}

So far as I've been reading Amos, I've been confronted with a persistent
question: Is it really possible for God's chosen people to fall out of his
favour? Motyer says that \enquote{We have forgotten that our God can turn and
become our enemy}\autocite[79]{motyer:2011}. Now, Motyer is addressing
Christians which I'll address in more detail below, under \textit{Application},
but in terms of Amos' audience, they would almost certainly have resisted the
idea that God would judge and punish his chosen people. They relied on the
promises they had inherited, the promises to Abraham, for their security.

The second of the two interpretations I listed above -- that Amos is driving
home a dreadful truth -- makes the most sense to me.

In verse 2, we can see that this is Amos' message: \enquote{\enquote{You only
have I chosen / of all the families of the earth; / therefore I will punish you
for all your sins.}} To paraphrase, I think Amos is saying, \enquote{Yes, you
are God's people, but do not think that this means you can't fall out of his
favour.} This is a difficult truth for the people to accept. Verse 3 is Amos
saying that, though the Israelites may think themselves protected no matter what
they do, God and Israel cannot walk together if the agreement that is the basis
of their relationship is ignored. In verses 4--6, Amos points out the universal
truth that actions have consequences and unfaithfulness to the covenant is no
exception. After reminding the people how cause and effect works, verses  6 and
8 apply this truth to the present unwelcome message: the trumpets have sounded,
the lion has roared, this is from the \textsc{Lord}, you should be afraid.

\subsection{What Hope is There?}

In chapter 2, God is reported as saying \enquote{For three sins of Israel, even
for four, I will not relent} (2:6). In chapter 3, God says \enquote{I will
punish you for all your sins} (3:2). Is there really no hope?

Calvin thinks there is hope, that God is not announcing the punishment for no
reason, but that so \enquote{that they might in time repent}\footnote{%
    \enquote{Then he shows that God designedly announces the punishment he would
    inflict on transgressors, that they might in time repent, and that he does
    not cry out for no reason, as unreflecting men grow angry for nothing, but
    that he is driven to anger by just causes and therefore terrifies them by
    his prophets.}
    \autocite[203-204]{calvin+owen:1986}
}. Motyer sees the question in v. 3 as not assuming the answer \enquote{No}, but
left intentionally unanswered with the following questions showing an interim
period of hope between initial cause and final effect\footnote{%
    v. 4:
    \begin{inparaenum}[(a)]
    \item the lion has his prey,
    \item the lion has caught his prey;
    \end{inparaenum}
    v. 5:
    \begin{inparaenum}[(a)]
    \item the bird is enticed by the bait,
    \item the trap has caught the bird;
    \end{inparaenum}
    v. 6:
    \begin{inparaenum}[(a)]
    \item the trumpet sounds as a warning,
    \item the disaster comes. \autocite[71]{motyer:2011}
    \end{inparaenum}
}. This \enquote{moment of hope} does not, however, \enquote{tarry
indefinitely}\autocite[71]{motyer:2011}. Smith says that \enquote{the revelation
of God's decisions to a prophet does not limit God's sovereign freedom to take a
contrary course of action if people repent}\autocite[152]{smith:2017}.

I think the answer lies in my general conclusions about these verses: Amos is
driving home an unwelcome truth, that God's judgement is a real possibility for
the covenant community of Israel. Amos' task is to bring this truth to bear, to
explain to the Israelite people that God's promises don't mean they can do
whatever they want. There does seem to be hope: we're in the interim period
where we've heard the lion roar but he hasn't yet attacked his prey. But I agree
with Motyer that the period of grace won't last for ever and so whether God's
judgement of the Israelites delivered through Amos was final with no hope of
repentance or the door was left open for repentance in chapter 3, we still have
to deal with the difficulties: there is a limit to God's grace toward members of
his people who knowingly and willingly flout the covenantal relationship and its
requirements. God will not walk endlessly with those who ignore their agreement
with him.

\subsection{Verse 12}

\VerseQuoteStyle
\begin{quote}
    This is what the \textsc{Lord} says:

    \VerseIndent\enquote{As a shepherd rescues from the lion's mouth\\
    \VerseIndent\VerseIndent only two leg bones or a piece of an ear,\\
    \VerseIndent so will the Israelites living in Samaria be rescued,\\
    \VerseIndent\VerseIndent with only the head of a bed\\
    \VerseIndent\VerseIndent and a piece of fabric from a couch.}
\end{quote}
\NormalQuoteStyle

You may read this as I did as referring to a \textit{remnant} that God would
save. Yes, the kingdom of Israel would be no more, but the few who were the true
Israel, the people of God would be saved. And while there is, perhaps, an
element of this in the verse\footnote{%
    Commenting on verse 12: \enquote{The nation will die and only a few dismal
    remnants of its once proud past will remain.}
    \autocite[386]{mccomiskey:2009}
}, there is an element that muddies the waters a bit: in Amos' time, if a sheep
was killed by a wild beast (such as a lion) the shepherd would recover whatever
he could as proof that he didn't steal the sheep himself\footnote{%
    \enquote{When a sheep was killed, the shepherd was required by Israelite law
    to recover the remains and return them to the owner: \enquote{If it is torn
    by a beast, let him bring it as evidence} (Exod. 22:10--13, [Heb 22:9--12])}
    \autocite[166]{smith:2017}

    \enquote{The same principle is found in the Code of Hammurabi, a
    Mesopotamian legal code about one thousand years before Amos.}
    \autocite[166, footnote]{smith:2017}
}. There are other parts of Amos, (\enquote{so that they may possess the
remnant of Edom and all the nations that bear my name}, 9:12a) that support the
remnant idea but it doesn't look like Amos is giving the Israelites this hope in
v. 12. On the contrary, it looks as though he's saying they'll be utterly
destroyed like a sheep by a lion and the customary evidence that this happened
will be all that remains.

\section{Application}

\subsection{Israel and the Church}

I have been thinking a lot about how the Old Testament should be read by
Christians, about what value it has outside of reporting the history of God's
acts in the world. What can we learn from Amos about who God is, what he has
done, who we are and how we should live in light of all this? Essentially, what
is the application of Amos for me, a Christian living in 21st century Australia?
In line with this, I've been trying to understand the relationship between
Israel and the church. I'm still very new to this, there's a lot more I need to
read and learn and think about before I even begin to approach conclusions, but
there are a few things that currently make sense to me and I think shed a good
amount of light.

As I've been thinking about this question, I've found N. T. Wright's
\textit{Christian Origins and the Question of God} series very helpful. In
particular the second and third volumes: \textit{Paul and the Faithfulness of
God} and \textit{The New Testament and the People of God}.

I'll offer a summary of some interesting points I found in these books related
to the relationship of Israel and the church and how I think that helps us read
and apply the Old Testament.

\begin{itemize}
    \item The Christian family is part of the Israelite family

        \begin{quote}
            From our earliest evidence, the Christians regarded themselves as a
            new family, directly descended from the family of Israel, but now
            transformed.
            \autocite[447]{wright:1992}
        \end{quote}

    \item This Christian family saw Israel's history as its own history

        \begin{quote}
            It [the eucharist] occupied a place similar to that of Passover in
            the Jewish community, except that it was celebrated not just once a
            year, but at least every week, reflecting the regular celebration of
            Jesus' resurrection on the first day of the week. It thus tied the
            life of early Christianity very firmly to the historical life of
            Israel. \dots We know of no early eucharist that did not recite the
            events of Jesus' death, much as the Passover liturgy recited the
            events of the exodus. Eucharist, like baptism, tied together
            continuity with Jewish history, an implicit claim,
            \textit{vis-\`{a}-vis} the pagan world, and Jesus.
            \autocite[448]{wright:1992}
        \end{quote}

        \begin{quote}
            \dots what united early Christians, deeper than all diversity, was
            \textit{that they told, and lived, a form of Israel's story which
            reached its climax in Jesus and which then issued in their
            spirit-given new life and task.}
            \autocite[][456, emphasis original]{wright:1992}
        \end{quote}

    \item Christians see Jesus and his people as replacing the Temple, the
        Torah, the Land and the City

        \begin{quote}
            No new Temple would replace Herod's, since the real and final
            replacement was Jesus and his people. No intensified Torah would
            define this community, since its sole definition was its
            Jesus-belief. No Land claimed its allegiance, and no Holy City could
            function for it as Jerusalem did for mainline Jews; Land had now
            been transposed into World, and the Holy City was the New Jerusalem.
            \autocite[451]{wright:1992}
        \end{quote}

    \item Jesus is the fulfilment of Israel's hopes

        \begin{quote}
            \dots when some hitherto frightened and puzzled Jews came to the
            conclusion that Israel's hope, the resurrection from the dead, the
            return from exile, the forgiveness of sins, had all come true in a
            rush in Jesus, who had been crucified.
            \autocite[452]{wright:1992}
        \end{quote}

    \item Christians saw themselves as the continuation of Israel, inheriting
        Israel's task

        \begin{quote}
            Those who now belonged to Jesus' people were not identical with
            ethnic Israel, since Israel's history had reached its intended
            fulfilment; they claimed to be the \textit{continuation of Israel in
            a new situation}, able to draw freely on Israel-images to express
            their self-identity, able to read Israel's scriptures and apply them
            to their own life. They were thrust out by that claim, and that
            reading, to fulfil Israel's vocation on behalf of the world.
            \autocite[457, emphasis original]{wright:1992}
        \end{quote}

        \begin{quote}
            Israel is called to a task (in the words of a learned Jewish
            correspondent, echoing centuries of tradition) of \enquote{repairing
            the world in God's name}.
            \autocite[775]{wright:2013}
        \end{quote}

        \begin{quote}
            Israel is God's \textit{servant}; and the point of having a servant
            is not that the servant becomes one's best friend, though that may
            happen too, but in order that, through the work of the servant, one
            may \textit{get things done}. And what \textsc{yhwh} wants done, it
            seems, is for his glory to extend throughout the earth, for all
            nations to see and hear who he is and what he has done.
            \autocite[805]{wright:2013}
        \end{quote}

        \begin{quote}
            The particular calling of Israel \dots would seem to be that
            \textit{through} Israel, the creator God will bring his sovereign
            rule to bear on the world.

            \dots since the claim is then made that this vocation, of Israel
            being a \enquote{light to the nations}, has been fulfilled in Jesus
            the Messiah. \dots I think that is exactly what Paul is talking
            about.
            \autocite[805-806]{wright:2013}
        \end{quote}

        \begin{quote}
            Granted the universality of human sin, as highlighted by the
            \enquote{apocalypse of the wrath of God} in the gospel (Romans
            1:18--2:16), what is to be done? Step forward the faithful Jew: this
            is the task of Abraham's family, to be the people through whom all
            this would be put right.
            \autocite[811]{wright:2013}
        \end{quote}

    \item Jesus and his followers were the true Israel in a way that national or
        ethnic Israel weren't

        \begin{quote}
            Following our exposition \dots it should be clear that texts which
            speak of the \enquote{coming of the son of man on a cloud} have as
            their obvious first-century meaning the prediction of vindication
            for the true Israel. Furthermore, from the use of these texts in the
            synoptic gospels it should be clear that the early Christians
            believed that Jesus was taking the place of that true Israel.
            \autocite[461]{wright:1992}
        \end{quote}

        \begin{quote}
            Within fifty years of the death of Jesus, by the time that Ignatius
            and Akiba were young men, those who saw themselves as Jesus'
            followers were claiming that they were the true heirs of the
            promises made by the creator god to Abraham, Isaac, and Jacob, and
            that the Jewish scriptures were to be read in terms of a new
            fulfilment.

            \dots Christianity's claim to be the true tenants of the vineyard
            was, naturally, resented.
            \autocite[467]{wright:1992}
        \end{quote}

    \item Paul's (i.e., Christian) theology doesn't break with Jewish theology,
        it revises or reapplies it

        \begin{quote}
            \dots the hypothesis at the heart of this book is that Paul's
            thought is best understood in terms of the revision, around Messiah
            and spirit, of the fundamental categories and structures of
            second-Temple Jewish understanding; and that this
            \enquote{revision}, precisely because of the drastic nature of the
            Messiah's death and resurrection, and the freshly given power of the
            spirit, is no mere minor adjustment, but a radically new state of
            affairs, albeit one which had always been promised in the Torah,
            Prophets and Psalms. The radical newness, then, does not alter the
            fact that Paul's theology is still a \enquote{revision} of Jewish
            theology, rather than a scheme drawn from elsewhere.
            \autocite[783]{wright:2013}
        \end{quote}

    \item In addition to there being no break in theology from Israel to the
        Church, there is also no historical break

        \begin{quote}
            There is, however, a phenomenon which is alive and well today,
            including in some prestigious places, which we might call
            \enquote{sweeping supersessionism}. This is the sweeping claim \dots
            that what happened in Jesus Christ constituted such a radical
            inbreaking or \enquote{invasion} into the world that it rendered
            redundant anything and everything that had gone before --
            particularly anything that looked like \enquote{religion}, not least
            \enquote{covenantal religion}. This view is unlike \enquote{hard
            supersessionism} because it denies that there is any historical
            continuity at all; it isn't that \enquote{Israel} has
            \enquote{turned into the church}, but rather that Israel, and
            everything else prior to the apocalyptic announcement of the gospel,
            has been swept aside by fresh revelation
            \autocite[807]{wright:2013}
        \end{quote}
\end{itemize}

There is a lot more to Wright's arguments and I encourage anyone who is
interested to learn more about this topic to read his books. Nevertheless, the
excerpts I have included above are, I think, sufficient to show that Paul and
other early Christians did not hold to the \enquote{sweeping supersessionism}
Wright mentioned. On the contrary, Christians were part of God's family,
Christians held Israel's history to be their own history, Christians saw Jesus
as the fulfilment of Israel's hopes (and the real Temple), Christians saw
themselves as the continuation of Israel (with Jesus as the true Israel) and
inheriting Israel's task. Not only is the Old Testament useful for us to
understand our history and we need to reject the idea that Jesus has
\enquote{rendered redundant anything and everything that [has] gone before}.
Israel's \enquote{covenant religion} is the Christian's religion.

As I read Amos 3, I've been thinking of some interesting parallels between the
Israel of Amos' day and the church. There was a national Israel, filled with
people who thought they would inherit the benefits of the covenant without
living in relationship with God (they held on to the promises); in a similar
way, there is a visible church, filled with people who think they will inherit
the benefits of Christ's work but who don't have a living relationship with God
through Christ. And there was a real, or true Israel (Amos, one assumes, was a
member of this true Israel) that both inherited the promises of the covenant and
took its requirements seriously, living in relationship with God; in a similar
way, there is the invisible church, the true church of God, whose members are
truly \textit{in Christ}, who take the requirements of the Christian life
seriously and who begin to benefit from Christ's work now and will receive the
full benefit in the future.

In light of this, I think we need to take what we've learnt from Amos about how
God deals with Israel and, as we worship the same God who has not changed,
realise that God can deal with the church in the same way. Just as the fact that
national Israel (technically though not practically) was God's people, known by
him did not protect them from the punishment of serious persistent apostasy, the
visible church cannot claim the promises of Christ as somehow protecting it
against the righteous judgement of God against serious sin. Amos threatened
Israel with destruction -- the nation of Israel was to be no more; there was
never a threat of the true Israel being destroyed (we know that God had a plan
to finally rescue his people through Jesus). Similarly, the visible church may
be destroyed, its buildings may be destroyed, its hierarchy disbanded, its
influence on society diminished, its reputation ruined -- this may be the
righteous judgement of God against those who claim to be his people but live in
total opposition to everything this means. The true Israelite need not despair
at Amos' words, at the possibility of future judgement against the nation of
Israel; the true Christian need not despair at anything that happens to the
visible church.

Despair isn't needed, but fear and trembling is. When Amos delivered this
judgement against Israel we're told \textit{fear} was the appropriate response
and those who followed God and listened to his prophets would have been jolted
out of any complacency they had when Amos delivered his message. The same is
true for us today, when the forces of this world seem to be working against the
church I think we should be asking ourselves if it's truly God working against
the church and, if it is, we should tremble with fear, we should be jolted out
of any complacency we have and we should repent.

\begin{quote}
    We who follow in the footsteps of Christ may also be tempted to the terrible
    error of an idolatry---be it illicit sex, money, power, or false doctrine.
    If we fall to any of these, we may need severe chastisement to advance our
    sanctification. How much better to flee from them (1 Cor. 6:18; 1 Tim. 6:11;
    2 Tim. 2:22) and to put on the full armor of God (Eph. 6:10--18).
    \autocite[388-389]{mccomiskey:2009}
\end{quote}

\begin{quote}
    Tradition taught that Israel was elect. God elected and redeemed his people
    from Egypt, but Amos announced a contrary message of punishment and
    destruction. This paradoxical reaction by God is fundamentally rejected by
    Amos' audience because it is contrary to their orthodoxy. \textit{This
    happened because their orthodoxy eliminated the dynamic of a trusting
    relationship with God and substituted a static non-relational guaranteed
    benefit based on covenant promises of blessing.}
    \autocite[154, emphasis mine]{smith:2017}
\end{quote}

Smith says that the \textit{orthodox} belief had \enquote{eliminated the dynamic
of a trusting relationship with God} and replaced it with a \enquote{static
non-relational guaranteed benefit}. Honestly, I think this is happening in the
church today. I don't think it's limited to proponents of so-called
\enquote{free grace theology}, I think it's present anywhere people teach that
the initial moment of salvation is the main event (whether they teach it as an
explicit doctrine or it's just a belief borne out in their practice) and that
lifelong sanctification (that requires continual, active engagement in
relationship with God) is a sideshow at best.

\subsection{Falling Out of God's Favour}

Amos paints us a picture of Israel: they are opulent, they have false religion,
they trust in promises instead of living in relationship, they have forgotten
how to distinguish right from wrong. Things have gotten so bad that the two
witnesses Amos calls, the wealthy of Ashdod and Egypt (v. 9) were from nations
Amos had just announced God's judgement against (Ashdod, cf. 2:8) and Israel's
historical enemy (Egypt, from whom God rescued them, cf. 2:10). Amos seems to be
saying that, as bad as the other nations are, Israel is worse; as deserving of
judgement as the other nations are, they still have more of a grip on what it
means to do right than does Israel.\footnote{%
    I'm not sure if this is simply a different usage of language or if Calvin
    sees something that I don't see in these verses, but he refers to the people
    in v. 9 as \textit{judges}: \enquote{Amos begins here to set judges over the
    Israelites; for they would not patiently submit to God's judgment: and he
    constitutes and sets over them as judges the Egyptians and Idumeans.}
    \autocite[213]{calvin+owen:1986}

    Smith points out that calling Egypt and Ashdod satisfies the requirement for
    two witnesses we see in Deuteronomy 17:6 and 19:15.
    \autocite[161]{smith:2017}
}

I don't think it's drawing too long a bow to connect this with some of the ways
in which society has called the church to account in recent years. Truths of God
have been twisted to support racism, oppression, hatred, greed. Some church
elders and leaders have used their positions to sexually abuse people while
others have acted to protect abusers or simply turned a blind eye to the
situation. The church has, I think, acted far too passively when it comes to
environmental concerns; we have ignored the task God has given us. There are
doubtless further evils and abuses to be uncovered and God is using people who
are not his children to bring justice where his children have brought injustice.
I wonder what Amos would say to this.

\begin{quote}
    Pusey recalls the imaginative way in which Cyprian, commenting on this
    passage, calls on \enquote{Jews, Turks and all Hagarenes} to behold the sins
    of Christendom: \enquote{\enquote{\dots a world reeking with mutual
    slaughter; and homicide, a crime in individuals, called virtue when wrought
    by nations} \dots immortal man glued to passing, perishable things! Men,
    redeemed by the blood of Jesus Christ, for lucre wrong their brethren,
    redeemed by the same price, the same Blood! No marvel then that the Church
    is afflicted, and encompassed by unseen enemies and her strength drawn down
    from her spoiled houses.} The only way in which this striking comment needs
    to be made appropriate to the present day of the church is that our enemies
    are no longer unseen.
    \autocite[83]{motyer:2011}
\end{quote}

As I quoted above, Alec Motyer says this:

\begin{quote}
    We have forgotten that our God can turn and become our enemy (Is. 63:10) and
    with all our talk of taking care not to fall into the power of Satan we have
    become blind to the much more dangerous possibility of falling out of the
    power of God.
    \autocite[79]{motyer:1974}
\end{quote}

Motyer refers to Isaiah 63:10:

\begin{quote}
    Yet the rebelled\\
    \VerseIndent and grieved his Holy Spirit.\\
    So he turned and became their enemy\\
    \VerseIndent and he himself fought against them.
\end{quote}

If this is true, it is certainly an inconvenient truth. One I'd much rather
forget, which is how I expect Amos' audience felt about his message. Motyer goes
on to say that

\begin{quote}
    Nothing which has taken place in Christ renders this truth void. We are
    God's covenant people, subject to His covenant blessings or His covenant
    curses. A study of the Letters of Jesus to the Seven Churches (Rev. 2:1 ff.)
    is particularly revealing in this connection (\textit{e.g.} 2:5, 16, 23;
    3:3, 16).
\end{quote}

When I first read Amos 3 and started thinking about the relationship between
Israel and the church, several things came to mind:
\begin{inparaenum}[(1)]
\item Jesus' words to the churches in Revelation 2--3, and
\item Jesus' words about true and false disciples in Matthew 7
\item the parable of the wheat and the tares in Matthew 13
\end{inparaenum}

\begin{itemize}
    \item In Matthew 7, we learn that there are many who will believe they
        belong to Christ who don't, to whom Jesus will say \enquote{Away from
        me, you evildoers!} (Matthew 7:23b).
    \item In Matthew 13, Jesus reinforces this same truth and warns us not to
        attempt to pull out the weeds as \enquote{[we] may uproot the wheat with
        them} (Matthew 13:29b).
    \item Finally, in Revelation we're told that if the Ephesian Christians do
        not repent, Jesus \enquote{will come to [them] and remove [their]
        lampstand from its place} (Revelation 2:5). In Revelation 1:20, we are
        told that \enquote{the seven lampstands are the seven churches}. Jesus
        seems to be saying that it's possible for Christians to lose their
        ability to be a church.
\end{itemize}

I haven't done any serious study on these passages and I'm sure my understanding
will mature, but, in light of both Amos and Jesus' words in Matthew and
Revelation I think Motyer was right when he said that we can become enemies of
God. And, if you look at the big picture of the church (at least in the wealthy
west), I think it looks frighteningly similar to the Israel of Amos' day.

I'll leave you with this quote:

\begin{quote}
    When the people of God behave towards each other as though the seamless robe
    of fellowship in Christ did not matter, when they are careless about social
    justice and welfare, when they lord it, insensitively, over others in things
    great or small, when they forget about a personal walk with God,
    Bible-reading, prayer, the fellowship of other believers, the Lord's Table,
    testimony to Jesus---these are rebellions, disobediences, contradictions of
    the known will of God for our lives. And there is no point in expecting
    anything but powerlessness and the adversity of an alienated God as long as
    we tarry in the place of rebellion.
    \autocite[85]{motyer:2011}
\end{quote}


\newpage

\thispagestyle{first}

\mbox{}\vspace{1in} % Extra margin

\printbibliography[title={BIBLIOGRAPHY}]

\end{document}

