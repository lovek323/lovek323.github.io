\newcommand{\Date}{June 22, 2019}
\newcommand{\Title}{Book Notes: \textit{Life Together} (Chapter 1)}

\documentclass[a4paper,12pt]{article}

% Lwarp
\usepackage[%
    latexmk,
    mathjax,
]{lwarp}

% Bibliography
\usepackage[style=sbl,fullbibrefs]{biblatex}
\addbibresource{../../bib/books.bib}
\addbibresource{../../bib/commentaries.bib}
\addbibresource{../../bib/commentary-series.bib}
\addbibresource{../../bib/dictionaries-encyclopedias.bib}
\addbibresource{../../bib/general-series.bib}
\addbibresource{../../bib/journal-articles.bib}
\addbibresource{../../bib/websites.bib}

\usepackage[bookmarks,hidelinks]{hyperref}
\usepackage[english]{babel}
\usepackage[margin=1in]{geometry}
\usepackage[multiple]{footmisc}

\usepackage{graphicx}
\usepackage{paralist}
\usepackage{times}

% Quotes
\usepackage[autostyle,english=british]{csquotes}
\DeclareQuoteStyle[american]{english}
    {\textquotedblleft}
    [\textquotedblleft]
    {\textquotedblright}
    [0.05em]
    {\textquoteleft}
    {\textquoteright}
\DeclareQuoteStyle[american-verse]{english}
    {\textquotedblleft}
    {\textquotedblright}
    [0.05em]
    {\textquoteleft}
    {\textquoteright}
\DeclareQuoteStyle[british]{english}
    {\textquoteleft}
    [\textquoteleft]
    {\textquoteright}
    [0.05em]
    {\textquotedblleft}
    {\textquotedblright}
\DeclareQuoteStyle[british-verse]{english}
    {\textquoteleft}
    {\textquoteright}
    [0.05em]
    {\textquotedblleft}
    {\textquotedblright}
% British
\newcommand{\OpenQuote}{\textquoteleft}
% American
% \newcommand{\OpenQuote}{\textquotedblleft}

% Lengths
\setlength{\parindent}{0pt}
\setlength{\parskip}{0.5\baselineskip}

% Poetry
\newcommand{\VerseQuoteStyle}{\setquotestyle[british-verse]{english}}
\newcommand{\NormalQuoteStyle}{\setquotestyle[british]{english}}
\newcommand{\VerseIndent}{\hspace*{2em}}
\newcommand{\VerseIndentTwo}{\hspace*{4em}}
\newcommand{\VerseIndentFour}{\hspace*{8em}}

\title{\Title}
\author{Jason O'Conal}
\date{\Date}

\begin{document}
\maketitle

\tableofcontents

\printbiblist{abbreviations}


\section{New Series: Book Notes}

I really love reading, but I often find that I can't remember what I've read
or I don't really \emph{learn} a lot from it. My experience in all areas of my
life is that things are far more likely to make an impression on me if I collect
my thoughts with a view to sharing them with others in such a way that they
might have a hope of understanding what I'm thinking.

There's a concept called \enquote{rubber-duck debugging}, where, \enquote{a
programmer would carry around a rubber duck and debug their code by forcing
themselves to explain it, line-by-line, to the duck}
\autocite{wikipedia:rubber-duck-debugging}. It's similar to the Latin proverb,
\textit{docendo discimus}, \enquote{by teaching, we learn}, which is apparently
the motto for various universities
\autocite{wikipedia:docendo-discimus}. The idea is basically that when you try
to explain something to someone who may not have the same level of
understanding, you have to explain any assumptions you've made and, when you do
this, you can often spot holes in them. At the very least, you realise that the
evidence in favour of your argument or your line of reasoning may not have been
as strong as you thought it was. I've found this to be regularly helpful to me
in work and the rest of life.

I've hit a bit of a dry spell in Amos. I've done the lion's share of my research
for the next post on Amos (creatively entitled \textit{Amos 4}), but I'm not
seeing anything particularly compelling coming out of it. Usually this just
means I need to take a break, so that's what I'm doing. I've got a long list of
books I'd like to read and I keep finding new ones\footnote{%
    I've added an incomplete \enquote{Reading List} page to this site:
    \url{https://oconal.id.au/reading-list.html} where I'll add new books as I
    find them.
}. It feels like a good time to tackle one that's been on my list for a few
years: \textit{Life Together} by Dietrich Bonhoeffer.

These posts will be different from my \textit{Bible Notes} series. The Bible
Notes posts are somewhat intentionally organised (even though it may not seem
like it), with a few ideas that I want to convey and explore. The Book Notes
posts won't be nearly as ordered. I'm not going to be writing book reviews or
summaries. As I read, I'll highlight things I found interesting or applicable to
me, my cultural context, or someone I knew and maybe jot down a quick note.
When I'm done with a chapter, I'll start writing it up. This write-up will
essentially be me including the most interesting or thought-provoking excerpts
and writing up my thoughts.

\section{%
    \textit{Life Together}
    \autocite{bonhoeffer:1996}
}

\textbf{Progress:} 47/140 pages

This book has been on my reading list for a long time. I've listened to an
audiobook version, which I really enjoyed, but there's no substitute for sitting
down and actually \emph{reading} the text (I have a tendency to miss a lot when
I'm just listening as opposed to reading).

Why did I choose to read this now?

For the last several months, I’ve been meeting with a few other Christians and
we’ve started exploring more informal ideas of church and Christian community.
Now, in some ways, \textit{Life Together} isn’t exactly the most obvious book
for this.  Bonhoeffer was writing about a seminary community where people were
training for ordination. Bonhoeffer himself was steeped in a rather formal
church tradition.  My community will look different from the one at Finkenwalde,
often vastly different. Nevertheless, one of the things I’ve been seeing more
and more clearly is just how high the calling of Christian community is for all
believers, not just those with certain titles or offices.

I have also got some questions around the Lord’s supper (the ‘love feast’ or
‘\textit{agape} feast’ of Jude 12), around communal/intercessory prayer, and on
Christian community in general. The final chapter of \textit{Life Together} is
entitled, ‘Confession and the Lord’s Supper’.

\section{Editor's Preface}

A great introduction by Geffrey B. Kelly, it added a lot of context I didn't
know about Bonhoeffer's life and theology, but it also raised a lot of
questions. I won't be able to draw any concrete conclusions of my own as to what
Bonhoeffer thought on any given topic, but I can still learn from what I read,
it can still give me some understanding.

\begin{quote}
    \dots to voice his conviction that the worldwide church itself needed to
    promote a sense of community like this if it was to have new life breathed
    into it.
\end{quote}

We are always talking about breathing new life into the church. Maybe this would
be Bonhoeffer's conclusion, that new life needs to be breathed into it, but I
wonder whether he would say we just need to recognise and rely on the life
that's already there.

\subsection{Life Together and the Crises of 1938}

I didn't highlight anything in this section, but it brought home to me just how
serious things were for true Christians at this time. Do we have any comparable
issues today?

\subsection{The Foundations of Bonhoeffer's Idea of Christian Community}

\begin{quote}
    He claimed in that first foundational study that `God's will is ever
    directed to the concrete human being.' In short, the will of God is
    expressed in a tangible word spoken to specific human beings and their
    communities. God's will should never be allowed to die the death of
    abstraction through its institutional, dogmatic, or biblicist reductionism.
\end{quote}

Earlier Kelly says, `He was guided by \dots the questions of how God in Christ
becomes present in and among those who profess faith in the gospel.' This is a
question I have too and I suspect it's a common one. The foundational study
being referred to here is \textit{Communion of Saints}, Bonhoeffer's treatise on
the church.

This sentence is a mouthful: `God's ``will'' should never \dots die the death of
abstraction through its institutional, dogmatic, or biblicist reductionism.'
What does it mean? I looked up `reductionism', and didn't learn anything
immediately obvious. To `abstract' something is to make it less direct, further
away, less concrete, practical, vibrant, personal, etc. For something to `die
the death of abstraction', then, could be that its personal, immediate, direct,
living character is minimised (or disregarded) in favour of something else. In
that case, perhaps Kelly is referring to the idea that people let institution,
dogma and types of biblical interpretation get in the way of God's `tangible
word spoken to specific human beings and their communities'. This may or may not
become clearer as I continue to read.

\begin{quote}
    The expression emanated from a deeply held conviction that Christian
    community had to integrate the gospel into its daily life and reflect this
    to the world. `Christ existing as community' challenges believers to behave
    as Christ to one another; this same Christ promises those who gather in his
    name to be present in, with, and for them.
\end{quote}

Just before this, Keller notes that Bonhoeffer's \textit{Sanctorum Communio}
defines the church as `Christ existing as community' and that `this was more
than a theological device', but (in this excerpt), a `deeply held conviction'.
Bonhoeffer had the same thoughts as many before him and after him: the gospel
must be part of the church's daily life and this is what we must reflect to the
world. He's wrestling with the question we saw before: `how God in Christ
becomes present in and among those who profess faith in the gospel'.

\begin{quote}
    His immersion in these projects yielded for him the conceptual grist for
    setting in motion a new way of being the church.
\end{quote}

Perhaps it was new to Bonhoeffer, to the Confessing Church, but nothing I've
read leads me to believe it was an entirely new way of being the church.

\begin{quote}
    At the inner core of the Christology that emerges from the Berlin
    dissertations is God's Word present in the human being Jesus and in the
    community of those with whom Christ identifies. \textit{Life Together} never
    strays from this form of Christocentrism.
\end{quote}

Again, we visit the question of what it means for Christ to dwell in the church.
It also ties in with the discussion of abstracting away the will of God. If
God's Word is present in the church (and it is living and active as we are told
in Hebrews 4:12; \textit{logos}, the same word as in John 1:1, though I'm not
sure how much that means), then we ought to be able to discern his will and
apply it to whatever situations we find ourselves in. I found it interesting
that Kelly didn't mention God's Word being present in individual Christians, but
`in the community of those with whom Christ identifies'; it's not even in the
community of those who identify with Christ, but those with whom Christ
identifies. It may not be worth making a big deal about, but the idea of the
corporate indwelling of Christ vs. the individual indwelling of Christ is
something I've been thinking about recently.

\begin{quote}
    Christ is depicted as the embodiment both of God and Christians, who are
    moved to do what, without Christ, they would be unable to accomplish: to
    live together, sharing faith, hope, and self-giving love in a prayerful,
    compassionate, caring community.
\end{quote}

This feels like one answer to the question we've been tracking. What does it
mean for Christ to dwell in the church? It means members of the church are
\textit{moved} to do what they would otherwise be unable to do: `to live
together, sharing faith, hope and self-giving love in a prayerful,
compassionate, caring community'. That sounds pretty amazing to me. Naturally,
as we get stuck into the actual contents of the book, we'll read more about
this, but the `self-giving love' becomes particularly important. I think I've
had glimpses of this sort of a community throughout my Christian life.

\begin{quote}
    The Christ of \textit{Life Together} is the binding force of that community
    in its `togetherness', gracing Christians to go beyond the superficial,
    often self-centered relationships of their everyday associations toward a
    more intimate sense of what it means to be Christ to others, to love others
    as Christ has loved them.
\end{quote}

\begin{quote}
    Bonhoeffer's entire approach to the community life experienced at
    Finkenwalde depends on a strong faith in the vicarious action of Christ in
    Word, sacrament, intercessory prayer, and service that makes it possible for
    Christians to be both `\emph{with} one another' [\emph{mit}einander] and
    `\emph{for} one another' [\emph{f\"ur}einander].
\end{quote}

This is a particularly interesting teaser. The `vicarious action of Christ' -- I
would love to see Christ operating through me more. What does he mean by `Word'?
Is it spending time in the Bible? Does it \emph{look} like spending time in the
Bible but actually go a lot deeper than that and involve a real meeting with the
`Word'? I am keen to understand these four things: Word, sacrament, intercessory
prayer and service. Maybe there are some things we're missing in our community?

\begin{quote}
    Bonhoeffer was unable to hide his aversion for attempts to etherialize the
    church into structures of empty ritual and perfunctory services that merely
    fronted for what purported to be an essentially `invisible heavenly
    reality'.
\end{quote}

There's a lot in this one. At this stage, I cannot know with a great degree of
confidence what Bonhoeffer was railing against. Here's my guess: churches where
Christians are simply performing the ritual, where `worship' becomes a thing
(similar to what we see in Amos and various other places) that has no
visible/experiential/practical impact on the `worshipper', the type of faux
inner devotion to the Lord Jesus that doesn't result in heart change, where we
no longer believe Jesus can and wants to be with us and lead us through our
lives in a personal way, where we just say the prayers or go through the motions
but we don't experience Christ's real presence in our churches. In case anyone
thinks I'm sitting here throwing stones at others, I know this to be a feature
of my own life. It's not even something just from the past, it's a trap I often
fall into. Part of the reason this book excites me so much is that I am
expecting to see some things I really need to see.

\subsection{Beginnings: The Search for Christian Community in Berlin}

\section{Preface}

\begin{quote}
    We are not dealing with a concern of some private circles but with a mission
    entrusted to the church. Because of this, we are not searching for more or
    less haphazard individual solutions to a problem. This is, rather, a
    responsibility to be undertaken by the church as a whole.
    \autocite[25]{bonhoeffer:1996}
\end{quote}

\section{Chapter 1}

\newpage

\thispagestyle{first}

\mbox{}\vspace{1in} % Extra margin

\printbibliography[title={BIBLIOGRAPHY}]

\end{document}

