\newcommand{\Date}{December 22, 2019}
\newcommand{\Title}{Course Notes: Jesus and the Gospels}

\documentclass[a4paper,12pt]{article}

% Lwarp
\usepackage[%
    latexmk,
    mathjax,
]{lwarp}

% Bibliography
\usepackage[style=sbl,fullbibrefs]{biblatex}
\addbibresource{../../bib/books.bib}
\addbibresource{../../bib/commentaries.bib}
\addbibresource{../../bib/commentary-series.bib}
\addbibresource{../../bib/dictionaries-encyclopedias.bib}
\addbibresource{../../bib/general-series.bib}
\addbibresource{../../bib/journal-articles.bib}
\addbibresource{../../bib/websites.bib}

\usepackage[bookmarks,hidelinks]{hyperref}
\usepackage[english]{babel}
\usepackage[margin=1in]{geometry}
\usepackage[multiple]{footmisc}

\usepackage{graphicx}
\usepackage{paralist}
\usepackage{times}

% Quotes
\usepackage[autostyle,english=british]{csquotes}
\DeclareQuoteStyle[american]{english}
    {\textquotedblleft}
    [\textquotedblleft]
    {\textquotedblright}
    [0.05em]
    {\textquoteleft}
    {\textquoteright}
\DeclareQuoteStyle[american-verse]{english}
    {\textquotedblleft}
    {\textquotedblright}
    [0.05em]
    {\textquoteleft}
    {\textquoteright}
\DeclareQuoteStyle[british]{english}
    {\textquoteleft}
    [\textquoteleft]
    {\textquoteright}
    [0.05em]
    {\textquotedblleft}
    {\textquotedblright}
\DeclareQuoteStyle[british-verse]{english}
    {\textquoteleft}
    {\textquoteright}
    [0.05em]
    {\textquotedblleft}
    {\textquotedblright}
% British
\newcommand{\OpenQuote}{\textquoteleft}
% American
% \newcommand{\OpenQuote}{\textquotedblleft}

% Lengths
\setlength{\parindent}{0pt}
\setlength{\parskip}{0.5\baselineskip}

% Poetry
\newcommand{\VerseQuoteStyle}{\setquotestyle[british-verse]{english}}
\newcommand{\NormalQuoteStyle}{\setquotestyle[british]{english}}
\newcommand{\VerseIndent}{\hspace*{2em}}
\newcommand{\VerseIndentTwo}{\hspace*{4em}}
\newcommand{\VerseIndentFour}{\hspace*{8em}}

\title{\Title}
\author{Jason O'Conal}
\date{\Date}

\begin{document}
\maketitle

\tableofcontents

\printbiblist{abbreviations}


\section{COURSE NOTES: JESUS AND THE GOSPELS}

Recently, I finished studying \textit{Jesus and the Gospels} online at Ridley
College. Each week, we were asked to submit a post on the discussion fora on a
particular topic. I've reproduced my posts here -- not because they're anything
special (or even remotely correct), but because I think it's often good to put
one's ideas out into the world. There's always the (slim) chance that someone
will engage with and improve them.

I can't reproduce the questions since I don't own them, but I think I'm fine to
share my own work. In addition, we were only allowed 250 words, so these will be
somewhat terse answers to unspecified questions. I have reproduced the week
titles from the Ridley course material.

\section{WEEK 1: MARK 1:1--15, SHAPE AND PURPOSE OF MARK}

\subsection{Issues of Genre}

Burridge noted several features of Graeco-Roman autobiography that are present
in Mark (continuous narrative prose, starts with the subject's public life, is
anecdotal, a large section given to the subject's death, the subject is the main
focus).\autocite[1-8]{burridge:2005} On the other hand, it is hard to deny that
Mark shares the theological message of Jesus as well as simply details about his
life. In Mark's Gospel I think we have the unique combination that Wenham and
Walton wrote about, the combination of autobiography (i.e., \enquote{lives}) and
a unique \enquote{gospel} genre that shares the message of what \enquote{God has
done in and through Jesus \dots\ in narrative
form}.\autocite[54]{wenham+walton:2011}

\subsection{Inklings of Purpose}

If it is true that Mark is the earliest Gospel and that it was written around 60
AD, it could have been written to preserve the eyewitness testimony that was
dying out due to old age. This is what Wenham and Walton refer to as the
\enquote{historical} reason for writing a
Gospel.\autocite[54-56]{wenham+walton:2011} They give three other reasons, two
of which I believe apply here: the \enquote{evangelistic} reason (we see this in
1:1, \enquote{The beginning of the good news about Jesus the Messiah, the Son of
God}) and \enquote{didactic} (quite a few of Jesus's teachings are recorded in
Mark).

\subsection{Ideas of Shape}

I see four sections to Mark's Gospel:

\singlespacing
\begin{enumerate}
    \item Beginnings (1:1--15)
    \item Ministry Part 1: Power, authority, healing (1:16--8:30) culminating in
        Peter's declaration that Jesus is the Messiah in 8:29
    \item Ministry Part 2: Suffering and death (8:31--15:47)
    \item Resurrection (16:1--8)
\end{enumerate}
\doublespacing


\section{WEEK 2: MARK 1:16--45, FOUR GOSPELS}

I have chosen to investigate the shape and purpose of the Gospel of John.

\subsection{Shape}

Wenham and Walton divide John's Gospel into two sections: Jesus's identity
(Messiah, Son of God, king) (1--12) and Jesus's death and resurrection
(13--21).\autocite[237]{wenham+walton:2011} Keener sees a prologue (1:1--18),
splits the second section (13:1--17:26 and 18:1--20:31), and adds an epilogue
(21).\autocite[420]{keener:2013} Borchert divides John into chapters 1--11 and
12--21.\autocites{borchert:1996}{borchert:2002} I favour Wenham and Walton's
structure to Keener's since they do not separate John's initial statements about
Jesus's identity. Wenham and Walton's inclusion of chapter 12 (anointing by
Mary, kingly entry into Jerusalem) in the first section seems preferable to
Borchert's as Jesus's kingship is a key part of his identity.

\subsection{Purpose}

John tells us his purpose in 20:31: \enquote{these are written that you may
believe that Jesus is the Messiah, the Son of God, and that by believing you may
have life in his name.} Borchert sees the primary purpose as evangelical, but
identifies several other purposes (apologetics, instruction, giving a sense of
time and purpose).\autocites[31--37]{borchert:1996} Keener says that, while some
see 20:31 as indicating an evangelistic purpose, \enquote{a large number of
scholars read this passage as a call for believers to persevere in
faith}.\autocite[428]{keener:2013} In John, the people recognise Jesus's
identity early (e.g., Andrew's declaration in 1:41), a theme that continues
throughout the Gospel, climaxing with Thomas's declaration that Jesus is
\enquote{My Lords and my God} (20:28). Jesus also declares who he is in the
\enquote{I am} statements (6:35; 8:12; 10:9; 10:11; 11:25; 14:6;
15:1).\autocite[429]{keener:2013} Thus, I see John's purpose as showing Jesus's
identity so that the reader might persevere in the belief that leads to eternal
life.


\newpage

\thispagestyle{first}

\mbox{}\vspace{1in} % Extra margin

\printbibliography[title={BIBLIOGRAPHY}]

\end{document}

