\newcommand{\Date}{February 14, 2021}
\newcommand{\Title}{Teaching: What Is the Bible?}

\documentclass[a4paper,12pt]{article}

% Lwarp
\usepackage[%
    latexmk,
    mathjax,
]{lwarp}

% Bibliography
\usepackage[style=sbl,fullbibrefs]{biblatex}
\addbibresource{../../bib/books.bib}
\addbibresource{../../bib/commentaries.bib}
\addbibresource{../../bib/commentary-series.bib}
\addbibresource{../../bib/dictionaries-encyclopedias.bib}
\addbibresource{../../bib/general-series.bib}
\addbibresource{../../bib/journal-articles.bib}
\addbibresource{../../bib/websites.bib}

\usepackage[bookmarks,hidelinks]{hyperref}
\usepackage[english]{babel}
\usepackage[margin=1in]{geometry}
\usepackage[multiple]{footmisc}

\usepackage{graphicx}
\usepackage{paralist}
\usepackage{times}

% Quotes
\usepackage[autostyle,english=british]{csquotes}
\DeclareQuoteStyle[american]{english}
    {\textquotedblleft}
    [\textquotedblleft]
    {\textquotedblright}
    [0.05em]
    {\textquoteleft}
    {\textquoteright}
\DeclareQuoteStyle[american-verse]{english}
    {\textquotedblleft}
    {\textquotedblright}
    [0.05em]
    {\textquoteleft}
    {\textquoteright}
\DeclareQuoteStyle[british]{english}
    {\textquoteleft}
    [\textquoteleft]
    {\textquoteright}
    [0.05em]
    {\textquotedblleft}
    {\textquotedblright}
\DeclareQuoteStyle[british-verse]{english}
    {\textquoteleft}
    {\textquoteright}
    [0.05em]
    {\textquotedblleft}
    {\textquotedblright}
% British
\newcommand{\OpenQuote}{\textquoteleft}
% American
% \newcommand{\OpenQuote}{\textquotedblleft}

% Lengths
\setlength{\parindent}{0pt}
\setlength{\parskip}{0.5\baselineskip}

% Poetry
\newcommand{\VerseQuoteStyle}{\setquotestyle[british-verse]{english}}
\newcommand{\NormalQuoteStyle}{\setquotestyle[british]{english}}
\newcommand{\VerseIndent}{\hspace*{2em}}
\newcommand{\VerseIndentTwo}{\hspace*{4em}}
\newcommand{\VerseIndentFour}{\hspace*{8em}}

\title{\Title}
\author{Jason O'Conal}
\date{\Date}

\begin{document}
\maketitle

\tableofcontents

\printbiblist{abbreviations}


\noindent As Seth mentioned, we're going to be spending a few weeks delving into
the idea of being formed by scripture. This series will go for several weeks and
we'll be hearing from Will and Seth as well in later weeks. This week, I've been
asked to talk about why God has given us the Bible. I'll start by looking at
what the Bible is.

\section{The Bible Is a Story}

I'll just spit it out: the Bible is a story. And that's probably something
you've heard several times before and, if you're anything like me, it didn't
mean a whole lot when you heard it. The team at the Bible Project love to talk
about how the Bible is one unified story that points to Jesus, so if you've
watched any of their videos, you've probably heard this quite a few times
before. But what does it mean to say that the Bible is a story?

\subsection{It's One Unified Story and It's Not About Us}

When we say that the Bible is a story, this means that it's a unified whole. You
can't just grab one bit of it and read it and expect it to make sense. Certainly
there are parts of it that make sense by themselves, but even then it's a
limited sort of sense.

It also means that it has the features of a story: it has a plot with a
beginning, a middle, and an end. Genesis tells us about the beginning,
Revelation tells us about the end, and the rest, the vast majority of the Bible,
tells us about the middle, which is the part of the story that we're living in.
This plot is actually retold again and again in miniature throughout the Bible:
God creates, humans rebel, there are consequences, God acts to restore, humans
rebel again, there are consequences again, and this continues so many times.

In addition to being a unified whole and having a beginning, a middle, and an
end, the Bible also has characters. \textit{But}, here's something we all need
to remember: humans are not the main characters of the Bible, people like
Abraham, Moses, David, and Paul are not the \textit{protagonists}, the ones that
drive the plot forward, people are usually the \textit{antagonists}, the ones
who get in the way of the plot moving forward, the ones who create the problems
that the main character has to find ways to fix. \textit{God} is the main
character. He's the one who exists at the beginning, is actively driving the
plot forward through the middle, and is ultimately victorious at the end.

We see throughout the Bible characters cast as new versions of previous
characters. Ezra might be seen as a new Moses, Nehemiah as a new Joshua, John
the Baptist as a new Elijah, Jesus as a new Adam, a new Moses, a new David, and
so on. One of the classic ways the Bible's story is told is the story of God's
rescue of his people from Egypt and Jesus's work is understood through this
story, he brings about the true and final exodus and he institutes a new
Passover with the once-and-for-all sacrifice for sin. In this way, we can see
that God is the main character, this time through Jesus -- Moses simply points
to Jesus, as do David, and Elijah, the whole Bible points to Jesus.

Another way we can know that the Bible is a story is in how many times the story
is retold within it. A great example is found in Nehemiah 9, which we looked at
a few weeks ago(Nehemiah 9:6--37). I won't read the whole thing, but it's like a
recap at the beginning of a new series of a TV show, or when they start a
two-parter with the classic, \enquote{Last time on\dots}. Nehemiah 9 starts with
creation and tells of God's actions through Abraham to the Exodus all the way to
the Babylonian exile from which they had just returned and it ends with a
description of God as a \enquote{mighty and awesome God, keeping covenant and
steadfast love} (Nehemiah 9:32). Nehemiah sees God as the main character in the
story, acting through all these times in all these different ways, and the
conclusion he draws is about God, not the Israelites.

The Bible is God telling us part of his story so that we can know him and what
he's like.

\subsection{It's One Unified Story and It Speaks to Those Who Listen}

There are a few verses that always seem to come up whenever Christians talk
about the Bible. I will not break from this tradition. Will someone read 2
Timothy 3:16--17 for us?

\begin{quote}
    All scripture is inspired by God and is useful for teaching, for reproof,
    for correction, and for training in righteousness, so that everyone who
    belongs to God may be proficient, equipped for every good work.
\end{quote}

This passage tells us a lot about the Bible, right? It tells us how it can help
us in our everyday walk with Jesus.

I read something this week that made me question an assumption I had in
interpreting this verse. It comes down to how I interpreted Paul's phrase
\enquote{all scripture} -- I thought that meant that each individual verse in
Scripture was useful \textit{by itself} for all these things. But when you
examine that idea, it becomes quite ridiculous.

To start with, chapters and verses are divisions that aren't actually part of
the inspired text (they were added later), so if we're going to take that idea
to the extreme, we'd have to say something like each individual \textit{word}
was useful in this way.  Words don't often have a lot of meaning by themselves,
so that can't be right.

So, do we instead take the approach that every \textit{book} is useful
\textit{by itself} for all these things that Paul mentioned? The books at least
are units of text that were inspired by God, right? That feels like it's getting
closer to something that might make sense. But we also know that God didn't give
us just one book of the Bible, he gave us the whole thing, in all its beautiful
diversity. Given this, reading a single book by itself without considering the
wider sweep of scripture doesn't seem right either.

Now, I think it's safe to say, that if all you had access to was, say, the
Gospel of Mark, you would find what you needed for salvation in there -- and it
would be very profitable for you to read that and internalise it -- but that's
not what this verse says, this verse isn't talking about something that
\textit{leads to} salvation, I think it's talking about how the Bible helps us
\textit{after}, once we have given our believing loyalty to king Jesus. In verse
15, just before this, Paul says that Timothy already knew the `sacred writings
that [were] able to instruct [him] for salvation' and yet Paul continued to talk
about what else was offered by \textit{all} scripture. Obviously, for Paul, the
Bible doesn't stop being useful for you once you're saved; on the contrary, it
becomes useful in a whole new way because it helps you live out the new life you
then have.

For example, how do we read Leviticus today as followers of Jesus? If I only
read Leviticus without understanding how Jesus read and interpreted it for his
followers, I might get stuck going down quite an unhelpful path. If I only read
Leviticus by itself without having also read Genesis and Exodus, I wouldn't see
the problem it's solving. And if I didn't continue on to read Numbers, I
wouldn't see that the solution provided by Leviticus actually worked. For the
modern reader, without this additional context, we're likely to come away from
Leviticus without seeing how well it tells us of God's rugged commitment to his
people despite all the difficulties that come from a holy God living with an
unholy people. If we read Leviticus without the rest of the Bible, we'd miss
God's revelation of his character that he wanted us to see. And if we ignore
Jesus's reapplication of parts of this text, we'll completely miss the fresh way
that God wants us to live it out. The same goes if we only read Jesus's words
and ignore the earlier parts of God's story; if we do that, we'll miss some of
the power and unexpectedness of Jesus's message.

Let me give you an example from a completely different context. Last year, I had
the pleasure of watching a TV show with Seth, called \textit{The Mandalorian}.
Now, I like \textit{Star Wars} as much as the next guy unless that next guy is
Seth. Seth understood things going on in that show that I never would have
picked up on because he's watched so many of the movies and other TV shows that
have come out in the \textit{Star Wars} universe multiple times and I've only
seen the movies once each. Every time Seth got excited about something, I paused
the show so he could tell me some awesome little Easter egg I'd missed or some
exciting thing that was likely to happen that I wouldn't have picked up on. It
made the experience so much more enjoyable for me. Watching \textit{The
Mandalorian} with Seth is a bit like reading the OT with Jesus and the other
authors of the NT, you get all this extra detail that you wouldn't have
otherwise known and you get so much more out of the experience. If I'd had
access to George Lucas, the writers of \textit{The Mandalorian} as well as a
bunch of people like Seth, I'd be starting to get close to what we have as
Christians when it comes to the Bible.

When we read the Bible, we not only have OT, the NT, what the NT says about the
OT, and what the church and the believing community throughout history has said,
we also have the very author of the text itself with us, personally speaking to
us about what we need to know to get the most out of it. It's a pretty amazing
deal, I must say. But to get everything God has for us, we need to engage with
all of the Bible, not just bits and pieces.

So the thing I realised as I was reading these verses in 2 Timothy may have been
obvious to everyone else, but the point is, this passage is not encouraging us
to look at individual parts of the Bible and look to them \textit{alone} for
teaching, reproof, correction, and training in righteousness, but to immerse
ourselves fully in the whole story of God, so that we can see who \textit{God}
is, where his heart is at, who \textit{we} are, and how we fit into his plan.
This is what I think Paul means by \textit{all scripture}. He means, read the
whole thing, don't stop just with the bits that help you find salvation because
there's so much more God wants to say than that. We do this, Paul says,
\textit{so that} we, who belong to God, may be \enquote{proficient, equipped for
every good work.}

\subsection{It's One Unified Story Written by Many Authors and a Single
Author}

The version that I quoted above, the NRSV, isn't that helpful for what I'm
about to say, so does anyone have a different version that they'd like to read
out?

In the NT times, \enquote{word of God} and \enquote{scripture} were not
necessarily the same thing. \enquote{Scripture} in 2 Timothy 3 translates the
Greek word \textit{graph\={e}}, which means \enquote{drawing, writing, painting,
or scripture}, but in various other places the \textit{word} part of `word of
God' is \textit{logos} (e.g., Hebrews 4). Sometimes \textit{logos} means the
spoken word, sometimes the written scriptures, sometimes Jesus's teaching, and
at other times, it refers to the divine nature and power of Jesus.
\textit{Graph\={e}}, on the other hand, specifically refers to the written
scriptures.

Now, in this verse, we're told that the \textit{written} scriptures are
\enquote{God-breathed} (\textit{theopneustos}, a word Paul seems to have
invented). Paul is saying that the source of our written scriptures is God's
very own breath, which is another way of saying that the written scriptures are,
in fact, also God's word.

This is a common pattern throughout the New Testament, let me give a couple of
examples:

\begin{itemize}
    \item Let's read Mark 12:36. Here we have Jesus quoting Psalm 110, saying
        that the Holy Spirit declared the content of that psalm through David.
        The Holy Spirit declared it! That's God speaking through the Bible.

        \begin{quote}
            \enquote{David himself, by the Holy Spirit, declared, / \enquote{The
            Lord said to my Lord, / \enquote{Sit at my right hand, / until I put
            your enemies under your feet.}}}
        \end{quote}
    \item In John 10:35, Jesus equates the \enquote{word of God}
        (\textit{logos}) with \enquote{scripture} (\textit{graph\={e}}). Jesus
        is telling us that \textit{scripture} is the \textit{word of God}.

        \begin{quote}
            If those to whom the word of God came were called
            \enquote{gods}---and the scripture cannot be annulled---
        \end{quote}
\end{itemize}

A Bible scholar, Ben Witherington III, says that \enquote{ancients did not think
words, and especially divine words, were mere ciphers or sounds. The ancients
believed words partook of the character and quality of the one who spoke them,
especially when talking about God's words.}

This felt like an important concept to grasp, but also pretty hard for me wrap
my head around and even harder to explain, but I'll try.

Okay, so you know how sometimes it feels like you get to know your favourite
authors even though you have probably never met them and maybe they're not even
alive? I love and have read a lot of C.S. Lewis's books and I feel like if I
heard a quote for the first time that I didn't know was his I'd have a pretty
good chance of guessing he's the author because I know what his writing sounds
like. If I continued to read more and more of his stuff, I might even get to the
point where I could guess what he might say about a topic that he never wrote
anything about. But I can't get too far because I have never met him in real
life and so I don't know his personality apart from what he's written.

The point I'm trying to make is this: scripture has both \textit{many} authors
and a \textit{single} author. And, while we don't get to meet the many human
authors, we do get to meet in an intensely personal way the one divine author,
the Holy Spirit. And in this way we can know God both through reading what he's
written and meeting him personally. And I think as we spend more time reading
the things God says in all the various ways he says them, we will get to know
his voice better. This is something Jesus promises us that we picked up on over
the last few weeks: he tells us that his sheep hear his voice in (John 10:27).

So, as we engage more with God through the Bible, we will learn what God's voice
sounds like more and more and we'll know when we're hearing it and when we're
not. So, for example, when Jesus commands his disciples to wash each others'
feet, do we believe that's a command we should follow literally today? If we
know God personally, we'll know how to answer that question. But we won't get to
this point, we won't learn to hear his voice if we don't spend time listening to
him through the scriptures. This could be by reading it ourselves, or it could
be by reading it together as a community, or it could be by listening to it on a
smartphone or in many other ways, the point is not that you have to sit
personally down and read the text day and night along with 50 commentaries, but
that when you read or hear it, you engage with it to build a listening
relationship with God and not simply to understand the written words.

When we read the scriptures we're not just reading marks on a page, when we hear
them read aloud, we're not simply processing vibrations in the air, the
scriptures we have before us are brought to life in our hearts, minds, and souls
by the Spirit of God.

\section{The Bible Is Alive, and It's Fiercely Powerful}

\begin{quote}
    We also constantly give thanks to God for this, that when you received the
    word of God that you heard from us, you accepted it not as a human word but
    as what it really is, God's word, which is also at work in you believers. (1
    Thessalonians 2:13)
\end{quote}

I said before that when we read the scriptures we're not simply reading marks on
a page, this verse in 1 Thessalonians 2 reaffirms that truth and makes it clear
that the Bible works not only on individuals, but also in communities.

We have the words of the living God, these words are powerful -- they are
inspired by the Spirit, they are living and active, they are at work in
believers. Now, I've said the words \enquote{living and active}, and that
probably reminds you of another classic verse on the Bible.

\begin{quote}
    Indeed, the word of God is living and active, sharper than any two-edged
    sword, piercing until it divides soul from spirit, joints from marrow; it is
    able to judge the thoughts and intentions of the heart. And before him no
    creature is hidden, but all are naked and laid bare to the eyes of the one
    to whom we must render account. (Hebrews 4:12--13)
\end{quote}

This verse is remarkable and it's worth pausing for a moment to consider what
it's saying. It's telling us that the word of God -- whether spoken or written --
actually does something to the hearer or reader. \textit{It does something!}
It's not simply that we \textit{consider} or \textit{pay attention to} its words
and perhaps change our minds, though that certainly happens. It's more than
that, it does something powerful and unexpected to the person who listens in
faith. I think this is something the Spirit does, this is part of the Spirit's
work as the promised guide and counsellor. If we are hearing or reading God's
word in \textit{relationship with God through the Spirit}, we'll find ourselves
changed and transformed, we'll find ourselves pierced, our thoughts and
intentions judged, our whole selves laid completely bare before our God and
judge. It's actually quite a sobering thought.

This is another affirmation of what we've been reading -- God's word, in all its
different aspects -- is alive and working in believers and believing communities
throughout the world. It's been this way since the Torah was first given to the
Israelites thousands of years ago and it will be this way at least until Jesus
returns.

\section{So, Why Has God Given Us the Bible?}

The simple answer is, I think, because God wants us to know him, his heart, what
he's done in the world, what he's done for us, and what he's going to do.

\section{Practice}

When we read the Bible, we should keep this in mind – the Bible is not a text
about \textit{how we should live}, it's a text about \textit{who God is} and its
primary purpose is to be a vehicle through which our relationship with God grows
and matures. As Christians, we don't so much have a relationship with the Bible,
but a relationship with God \textit{through} the Bible.

Read and study Psalm 23 this week by doing two things:

\begin{enumerate}
    \item Look up all the cross references, see how it's been used throughout
        the rest of God's story
    \item Write it in your own words as a prayer to Jesus, the Good Shepherd
\end{enumerate}

\newpage

\thispagestyle{first}

\mbox{}\vspace{1in} % Extra margin

\printbibliography[title={BIBLIOGRAPHY}]

\end{document}

