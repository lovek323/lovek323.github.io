\newcommand{\Date}{February 28, 2021}
\newcommand{\Title}{Teaching: What the Bible Is Not}

\documentclass[a4paper,10pt]{article}

% Lwarp
%\usepackage[%
    %latexmk,
    %mathjax,
%]{lwarp}

% Bibliography
\usepackage[
    citepages=omit,
    fullbibrefs,
    ibidtracker=false,
    idemtracker=false,
    style=sbl,
]{biblatex}
\addbibresource{../../bib/books.bib}
\addbibresource{../../bib/commentaries.bib}
\addbibresource{../../bib/commentary-series.bib}
\addbibresource{../../bib/dictionaries-encyclopedias.bib}
\addbibresource{../../bib/general-series.bib}
\addbibresource{../../bib/journal-articles.bib}
\addbibresource{../../bib/websites.bib}
\setlength{\bibhang}{0.5in}

% Packages with arguments
\usepackage[bookmarks,hidelinks]{hyperref}
\usepackage[rm,tiny]{titlesec}
\usepackage[english]{babel}
\usepackage[margin=1in]{geometry}
\usepackage[marginal]{footmisc}

% Packages without arguments
\usepackage{afterpage}
\usepackage{booktabs}
\usepackage{fancyhdr}
\usepackage{graphicx}
\usepackage{longtable}
\usepackage{paralist}
\usepackage{ragged2e}
\usepackage{scrextend}
\usepackage{setspace}
\usepackage{times}
\usepackage{tocloft}
\usepackage{xcolor}

% Quotation marks -- British style
\usepackage[autostyle,english=british]{csquotes}
%\DeclareQuoteStyle[american]{english}
    %{\textquotedblleft}
    %[\textquotedblleft]
    %{\textquotedblright}
    %[0.05em]
    %{\textquoteleft}
    %{\textquoteright}
%\DeclareQuoteStyle[american-verse]{english}
    %{\textquotedblleft}
    %{\textquotedblright}
    %[0.05em]
    %{\textquoteleft}
    %{\textquoteright}
\DeclareQuoteStyle[british]{english}
    {\textquoteleft}
    [\textquoteleft]
    {\textquoteright}
    [0.05em]
    {\textquotedblleft}
    {\textquotedblright}
\DeclareQuoteStyle[british-verse]{english}
    {\textquoteleft}
    {\textquoteright}
    [0.05em]
    {\textquotedblleft}
    {\textquotedblright}
% British
\newcommand{\OpenQuote}{\textquoteleft}
% American
%\newcommand{\OpenQuote}{\textquotedblleft}

% Poetry
\newcommand{\VerseQuoteStyle}{\setquotestyle[british-verse]{english}}
\newcommand{\NormalQuoteStyle}{\setquotestyle[british]{english}}
\newcommand{\VerseIndent}{\hspace*{2em}}
\newcommand{\VerseIndentTwo}{\hspace*{4em}}
\newcommand{\VerseIndentFour}{\hspace*{8em}}

% Lengths
\setlength{\parindent}{0.5in}
\setlength{\RaggedRightParindent}{\parindent}

% Line spacing
\doublespacing{}

% Justification
\RaggedRight

% Footnotes
\setlength{\footnotesep}{19.86pt} % Determined by running \footnotesize \the\baselineskip
\renewcommand{\footnoterule}{\noindent\smash{\rule[3pt]{2in}{0.4pt}}\vspace{-0.5\footnotesep}}
\setlength{\footnotemargin}{0.5in}
\deffootnote[0.5in]{0pt}{0.5in}{}
\let\TempFootnote\footnote
\renewcommand{\footnote}[1]{\TempFootnote{\thefootnotemark.\enskip#1}}
\setlength{\skip\footins}{\baselineskip}

% Table of contents
\renewcommand\cftsecfont{\rm}
\renewcommand\cftsecleader{\cftdotfill{.}}
\renewcommand\cftsubsecdotsep{\cftnodots}

% Word count
%TC:newcounter fwords Words in footnotes
%TC:newcounter footnote Number of footnotes
%TC:macro \footnote [fwords]
%TC:macroword \footnote [footnote]
%TC:macro \autocite [fwords]
%TC:macroword \autocite [footnote]

\begin{document}

% Sections
\setcounter{secnumdepth}{0}
% - Section
\titlespacing*{\section}{0pt}{\baselineskip}{\baselineskip}
\titleformat*{\section}{\center\uppercase}
% - Subsection
\titlespacing*{\subsection}{0pt}{\baselineskip}{0pt}
\titleformat*{\subsection}{\center\bfseries}

% Headers & Footers
\renewcommand{\headrulewidth}{0pt}
\fancypagestyle{first}{%
    \fancyhf{}
    \fancyfoot[C]{\thepage}
}
\fancypagestyle{subsequent}{%
    \fancyhf{}
    \fancyhead[R]{\thepage}
}
\pagestyle{empty}
\pagenumbering{roman}

\mbox{}

\vfill

\begin{center}
    \Huge{\textbf{\Title}}
\end{center}


\vfill

\begin{center}
    \Date

    Jason O'Conal
\end{center}

\vfill
\newpage

\pagestyle{plain}

\mbox{}\vspace{1in} % Extra margin

\renewcommand{\contentsname}{}
\section*{CONTENTS}
\tableofcontents

\vfill
\newpage

\mbox{}\vspace{1in} % Extra margin

\printbiblist[title=ABBREVIATIONS]{abbreviations}

\vfill
\newpage

\pagestyle{subsequent}

\thispagestyle{first}

\pagenumbering{arabic}

\mbox{}\vspace{1in} % Extra margin


\section{Introduction}

Two weeks ago, we looked at what the Bible \textit{is}. This week, I want to look at
what the Bible \textit{is not}.

There are three mishandlings of the Bible that I'd like to discuss:

\begin{enumerate}
    \item Viewing the Bible as primarily a lawbook
    \item Viewing the Bible as a collection of promises, blessings
    \item Viewing the Bible as a remote, impersonal authority
\end{enumerate}

\section{Lawbook}

Firstly, if we read the Bible as just a lawbook, we'll be constantly condemning
ourselves and others. This is, I think, my `safe place'. I like rules, I like
being told what to do, and I love telling others what to do. As I grow older and
experience my own sinfulness more fully, I'm becoming more and more thankful
that God didn't simply give us a list of rules or laws we had to follow.

Now, of course, I do think the Bible helps us understand how to live, it just
does it more through revelation of who God is than a list of rules. The first
five books of the Bible are referred to by various names, one of which is the
Law of Moses. These books have way more story than they do law. When God gave
the Israelites their law, he did it through story.

The Bible's primary role is to tell us about God and his story, not to give us a
once-and-for-all list of exact behaviours to pursue or avoid. It tells us about
a God who is both `merciful and gracious, / \dots forgiving iniquity and
transgression and sin, / yet by no means clearing the guilty' (Exodus 34:6--7).
Well, if the Bible is just a lawbook, then I am most certainly guilty and I've
no hope of forgiveness. But, of course, it's more than that.

\section{Promises and Blessings or Curses Taken Out of Context}

\subsection{The Problem of Contradicting Promises}

Secondly, the Bible is not a collection of promises, blessings, and curses that
can be ripped out of context.

Let me read a verse from Psalm 37:

\begin{quote}
    I have been young, and now am old, yet I have not seen the righteous
    forsaken or their children begging bread.
\end{quote}

Hmm, is that really how it works \textit{all the time}? I \textit{have} seen the
`righteous forsaken', or at least it looks like they were. I've seen horrible
things happen to followers of Jesus, I've heard stories of Christians living in
poverty. If we read the Bible in this way, we won't know how to handle it when
life doesn't live up to these promises or our understanding of them. Of course,
we could spiritualise this promise away, right? We could say, yeah, the
righteous will never be forsaken finally, they'll be with God and Jesus
eternally, but I think the second part of it doesn't really make sense in this
way, it's talking, like a lot of the wisdom literature in the Bible, about life
being good in the here and now.

Let me give another example. Take Proverbs 1:19 where it says `Such is the end
of all who are greedy for gain; / it takes away the life of its possessors'. I
have heard of plenty of greedy, selfish people who seem to prosper in this life,
who die rich and seemingly happy. Again, we could spiritualise this away, and
say they'll be punished eternally, but I don't think that's what it means and I
don't think it would've meant that to the original readers.

Another book in the Bible takes up the same question we have here. Here's what
it says in Ecclesiastes 7:

\begin{quote}
    In my vain life I have seen everything; there are righteous people who
    perish in their righteousness, and there are wicked people who prolong their
    life in their evil-doing. (Ecclesiastes 7:15)
\end{quote}

We have two verses that say seemingly contradictory things. Why is it that
sometimes terrible things happen to good people? Why is it sometimes really
positive things happen to bad people? If we take the promises and blessings
approach, we won't know how to deal with life when it doesn't follow the rules.

In Matthew 5, Jesus says that `[God] makes his sun rise on the evil and on the
good, and sends rain on the righteous and the unrighteous' (Matthew 5:45). In
John 9, when Jesus encountered the man born blind, he said that `Neither this
man nor his parents sinned; he was born blind so that God's works might be
revealed in him' (John 9:3). The man's blindness had nothing to do with promises
or blessings or curses; it had everything to do with God making a choice with a
purpose in mind.

\subsection{Scripture in Conversation with Itself}

Scripture is constantly in conversation with itself. I would love to talk more
about how wisdom literature works in the Bible, but I'll try to keep it really
brief. Proverbs contains small pithy sayings that strip out pretty much all the
extra qualifying information to boil down the statements to something memorable.
We do the same thing in English, like when we say `look before you leap'. We
know what it means and when it applies, but someone outside of our cultural
context might need to do a bit of work to understand it. Proverbs looks at
wisdom as a feature of how the universe generally works and says if you live
wisely in fear of the Lord, you're doing what you can to set yourself up for a
good life.

\begin{quote}
    It is neither selfish nor unrealistic for a parent to wish a child a
    reasonable level of success in life---including social acceptance, moral
    uprightness, and freedom from want. The book of Proverbs provides a
    collection of pithy advisory statements designed to do just that. There is
    no guarantee, of course, that a life will always go well for a young person.
    What Proverbs does say is that, all things being equal, there \textit{are}
    basic attitudes and patterns of behavior that will help a person grow into
    responsible adulthood.
\end{quote}

\begin{quote}
    Consider the English proverbs “Look before you leap” and “A stitch in time
    saves nine.” The repetition of single-syllable words beginning with the
    letter \textit{l} in the first case and the rhythm and rhyme of
    single-syllable words in the second case are elements that give these
    proverbs a certain catchiness. They are not as easy to forget as would be
    the following statements: “In advance of committing yourself to a course of
    action, consider your circumstances and options”; and “There are certain
    corrective measures for minor problems that, when taken early on in a course
    of action, forestall major problems from arising.” These latter formulations
    are more precise but lack the punch and effectiveness of the two well-known
    wordings, not to mention the fact that they are much harder to remember.
    “Look before you leap” is a pithy, inexact statement; it could easily be
    misunderstood, or thought to apply only to jumping. It does not say where or
    how to look, what to look for, or how soon to leap after looking, and it is
    not even intended to apply literally to jumping!
\end{quote}

Ecclesiastes is more experimental and raises the question, `Why don't things
always work the way we're told things are supposed work?' Job looks at a
specific situation where the conventional wisdom that good things happen to good
people and bad things happen to bad people just doesn't apply. When we read
these three books together, we find that it's really not as simple as a 100%
guarantee that if you are righteous you'll never have any trouble.

In fact, Jesus told us quite the opposite. In John 15, he said that `if they
persecuted me, they will persecute you' (John 15:20). We are told that, as
followers of Jesus, we will suffer persecution as he did. Though this, too,
isn't something we should take out of context. It's not as if a Christian who
isn't suffering persecution isn't really a Christian, it's more that persecution
isn't something Christians should be surprised by. Because it happened to the
one we follow, it's likely to happen to us, too.

\subsection{The Gospel and This Method}

The blessings and curses way of reading the Bible conflicts with the radical,
reckless nature of the Gospel. It leads us to think that if we \textit{do} the
right thing, we'll be happy, and if we \textit{do} the wrong thing, we'll be
sad. But that's not the message of the gospel, where even the vilest, most evil,
most degenerate person, is no better than the most righteous person. Nobody is
good except God, we all need the forgiveness that comes through Jesus's life,
death, and resurrection, and we're all equally saved if we turn to him for
salvation. How does that fit in to this way of reading the Bible?

And, if we do manage to find a way through all that, there'll be a danger of
becoming legalistic, pushing our own understanding of how the promises,
blessings, and curses work onto others. I'm thinking of, for example, what
happens with the so-called prosperity gospel. I was in a church service once
where someone asked the pastor, `Should we tithe based on our net or gross
income?' And the response was, `Well, it depends on whether you want net or
gross blessings.' Yes, it is more blessed to give than to receive (Acts 20:35),
but there isn't some direct line between the dollar amount that we give and the
value of the blessings we receive. We are already blessed beyond measure, and we
can't put God in our debt. It's more blessed to give than to receive, I think,
because giving \textit{without} wanting to receive in return frees us from the
clutches of materialism and makes us more like Christ.

If we do find ourselves walking down the legalistic path, we'll be taking the
job of the Spirit and the Bible for ourselves, putting ourselves above God and
his word. The Bible was never designed to work this way, it was never intended
to be a text that a person studies and then applies to everyone around them
without the involvement of God's Spirit. God has made it as something that he
wants us all individually and communally to be formed by.

\section{A Remote, Impersonal Authority}

\subsection{The Bible As Our Final Authority?}

Finally, on the third misuse of scripture, it is a common phrase among
evangelical Christians to talk about how the Bible is our final authority.
There's nothing wrong with this statement when it's understood properly, I've
said it myself and I like it a lot. It keeps us humble, it works against the
wrong idea that \textit{I} might be the final authority in my life. But, it's
missing something pretty crucial -- the Bible was never meant to be read outside
of a relationship of submission to God. We submit to \textit{God} not the Bible,
and one of the ways we submit to God is through reading scripture and allowing
it to shape us.

There are so many contentious issues on which genuine Christians think their way
of reading the Bible is the only correct way. It's great to have it as a final
authority, but what happens often -- and I've fallen victim to this probably
more than most -- is that we read our own theology into the Bible instead of the
other way around like it's supposed to work. Sometimes when people say `the
Bible is our final authority', they think that there's only one obvious way to
read and understand the Bible. This leaves us with Christians and churches who
claim to follow the same Jesus and have him as their king, but instead submit
only to their own interpretation of his word.

In some cases, when people say that the Bible is their final authority, they
mean their own, their church's, or their pastor's \textit{interpretation} of the
Bible is their final authority. And that results in something completely opposed
to what they think they're claiming -- they think they're claiming that they've
put themselves \textit{under} the word of God, but the reality is that they've
put themselves or someone else \textit{over and above} the word of God. And I
don't think for a second that I'm immune to this danger.

\subsection{Whatever You Think You Know, You'll Be at Least a Bit Wrong,
Probably a Lot}

There's something I really want us all to understand when it comes to being
formed by scripture: whatever you think about the Bible, whatever you think you
understand of what it teaches, whatever conclusions you've drawn, you're going
to be at least a little bit wrong. You'll miss some of the picture, it's just
too big for us to comprehend. And that is the same for everyone! There is no
person who can say they fully understand the Bible because the Bible isn't a
static thing that can be fully understood, it's a dynamic thing that's
understood and applied fresh each day by the Spirit in us and in our community.
You can't read it, work it out, understand it, and then say, `Cool, I know what
scripture says now, I'm good.'

There's a great book by Frank Viola called \textit{ReGrace} that looks at all
these famous Christians from history from all different parts of the church,
from church fathers to Martin Luther to Charles Spurgeon to C.S. Lewis and he
points out things that they believed or practiced that Christians today might
say are just plain wrong. His point is that nobody's interpretation of scripture
and understanding of God is perfect, we shouldn't claim that it is, and we
shouldn't vilify or demonise those who disagree with us.

In his famous work, \textit{Life Together}, Dietrich Bonhoeffer says this:

\begin{quote}
    Because they can no longer consider themselves wise, Christians will also
    have a modest opinion of their own plans and intentions. They will know that
    it is good for their own will to be broken in their encounter with their
    neighbor.  They will be ready to consider their neighbor's will more
    important and urgent than their own. What does it matter if our own plans
    are thwarted? Is it not better to serve our neighbor than to get our own
    way?
\end{quote}

\subsection{Dangers of This Method}

If we give only a few people in our community the authority to interpret the
Bible for us, we risk putting them above God in our lives, and that can be a
very dangerous thing. During our discussion two weeks ago, the sad truth was
brought up that certain ways of interpreting the Bible have led to abuse in the
past, and I think it's this third way that's most prone to that.

I recently finished reading another book by Scot McKnight called \textit{A
Church Called Tov} in which he quotes one of the elders of a church as saying
`publicizing viewpoints rejected by the elder majority for any reason is satanic
to the core'. Wow!

\begin{quote}
    Two former elders at Harvest were formally excommunicated by the church and
    demonized in a video explaining the decision to the congregation, after they
    persisted in questioning decisions made by the board. One board member went
    so far as to say that `publicizing viewpoints rejected by the elder majority
    for any reason is satanic to the core.'
\end{quote}

Just a brief look at church history shows how easy it is for people in authority
to stop listening to the Spirit and go astray. I think, had I lived during
Jesus's time, I would've wanted to be a Pharisee. I'm that kind of person. God's
people have often wrestled with how to interpret scripture. We have all the
different theological issues in the early church documented in the NT. We have
the debates around the nature of Jesus's divinity and the trinity that happened
in the early centuries AD. Fast forward to the 1,600s and we have the
rediscovery of the gospel of grace and the reintroduction of the Bible to
believers at large. Then there's the question of slavery. Every generation needs
a fresh movement of the Spirit to push us closer to the Father's heart. The
answers from our parents' generation don't work for us, we have to go to the
Spirit to get our own. The answers from our pastors don't work either, we have
to go to the Spirit to get our own.

It's easy to look back on the past and say, well, they clearly got that wrong.
But at all these different stages throughout church history, there were people
who thought they were undeniably right in their interpretation of scripture.
It's a dangerous thing to think that you're totally right and that the Spirit
won't move afresh to remind you of something you've missed and bring you back to
the Father's heart.

The Reformers had an idea about how the Bible was easy to understand, they
called it the `perspicuity of scripture'. Now, some people think this means that
the Bible is totally obvious in everything it teaches -- and, naturally, they
consider their own interpretation to be the obvious one. They get the idea that
if you have a different understanding, you're either deliberately missing the
point or you're spiritually blind. Now we know that's it's not as simple as that
and it's also not what the Reformers were saying. This idea of the Bible's
clarity that the Reformers pushed was talking about the same thing I'm talking
about today -- we don't need some spiritual authority like a pope, a bishop, a
priest, or a pastor to understand the Bible, we don't need a university degree.
As God's children empowered by the Spirit, we are all qualified to read the
Bible for ourselves. It's not a dangerous book that we need to be protected
from.

\begin{quote}
    The Protestant Reformers appealed to the principle of biblical clarity to
    show that believers are not bound to the clergy in studying Scripture, but
    that they are qualified to read the Bible for themselves. The clarity of
    Scripture does not mean that we need no teachers; Scripture says that God
    provides teachers to the church as part of his clear communication of the
    biblical content. But the doctrine of clarity calls on everyone---of
    whatever age, social status, or education level---to hear God's word.
\end{quote}

\subsection{A Path Forward}

So how do we live with God as our authority and the Bible as part of his
revealed will to us without falling prey to this trap? I don't have a foolproof
plan, I think ultimately we need to trust God and the Spirit that he will bring
us the correction we need as we go along, but I think we can try to stay on the
right path. We need to remember Jesus's words that those whom we might call
`leaders' or `teachers' in our community are not to rule over those in their
care, but to copy the example of our shepherd leader, guiding people with a
pastoral heart (Matthew 20:25--27).

\begin{quote}
    But Jesus called them to him and said, `You know that the rulers of the
    Gentiles lord it over them, and their great ones are tyrants over them. It
    will not be so among you; but whoever wishes to be great among you must be
    your servant, and whoever wishes to be first among you must be your slave;'
\end{quote}

This means that we shouldn't use the Bible to convince someone else of what God
has told us, of our own application of the Bible's message to our life and
situation. I think this is what people mean when they talk about Bible bashing
and I don't really think it works. What we can do is present what God has told
us and trust that the Spirit will confirm that in the hearts of those who are
listening if it truly is from him. If it's not from God, or it is from God but
it's not what God has to say to them at this moment, it's good if they don't
take it on.

Let me read a little more from that Bonhoeffer's \textit{Life Together} that
expresses this point really well.

\begin{quote}
    Because Christ stands between me and an other, I must not long for
    unmediated community with that person. As only Christ was able to speak to
    me in such a way that I was helped, so \textbf{others too can only be helped
    by Christ alone}. However, this means that \textbf{I must release others
    from all my attempts to control, coerce, and dominate them} with my love. In
    their freedom from me, other persons want to be loved for who they are, as
    those for whom Christ became a human being, died, and rose again, as those
    for whom Christ won the forgiveness of sins and prepared eternal life.
    Because Christ has long since acted decisively for other Christians, before
    I could begin to act, \textbf{I must allow them the freedom to be Christ's}.
    They should encounter me only as the persons that they already are for
    Christ. \dots

    Therefore, spiritual love will prove successful insofar as it commends
    Christ to the other in all that it says and does. It will not seek to
    agitate another by exerting \textbf{all too personal, direct influence or by
    crudely interfering in one's life}. \dots Rather, it will encounter the
    other with the clear word of God and \textbf{be prepared to leave the other
    alone with this word for a long time}. It will be willing to \textbf{release
    others again so that Christ may deal with them}. \dots This spiritual love
    will thus \textbf{speak to Christ about the other Christian more than to the
    other Christian about Christ}. It knows that \textbf{the most direct way to
    others is always through prayer to Christ}.
\end{quote}

If we're reading the Bible as story in relationship with God through the Spirit
we will all be challenged in our understanding of what it's saying because its
message is applied fresh each day to every one of our lives and God's message to
me isn't necessarily going to be God's message to you. I need God's fresh
message for me regularly because my heart's default position seems to have
nothing to do with what God wants. I need constant correction, usually in the
same areas. Throughout the Bible, God speaks to his people in their own day in
their own way. God didn't deal with Abraham in the same way he dealt with Moses
or David or Solomon or Jesus's disciples or Paul or the early Christians and he
doesn't deal with us in the same way he deals with our parents or Christians
even in the next church over.

\section{Conclusion}

What I'm saying is this: it is possible to read the Bible in a lot of different
ways, but I think God's designed it in a specific way. If you think you can
engage the Bible and be formed by it simply through reading it as a dry text,
listening to sermons, saying you believe it, you're misusing it and you're not
going to get what you're looking for. If, instead, you develop a
\textit{listening} relationship with God through the Bible, if you genuinely
place yourself \textit{under} his authority and allow the Spirit to challenge
and upset what you think you know and how you think you should behave -- and if
you're willing to let the Spirit do that same work in others without taking it
over yourself -- then you're just where God wants you and you'd better get ready
for a wild ride.

\newpage

\thispagestyle{first}

\mbox{}\vspace{1in} % Extra margin

\printbibliography[title={BIBLIOGRAPHY}]

\end{document}

